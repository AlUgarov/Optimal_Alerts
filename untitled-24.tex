\documentclass[11pt,hyperref={bookmarks=false}]{beamer}
\usetheme{Warsaw}
%\usetheme{Madrid}
%\usecolortheme{beaver}
\usefonttheme{professionalfonts}
% \usepackage[usenames,dvipsnames]{pstricks}
 %\usepackage{wallpaper}
 \usepackage{epsfig}
\definecolor{UniBlue}{RGB}{157,34,53}
\setbeamercolor{block title}{bg=UniBlue!70,fg=black}

\usepackage{psfrag,graphicx}
\usepackage{amsmath,amsfonts}
\usepackage{adjustbox}
\usepackage{lscape}
\usepackage{array,epsfig}
\usepackage{amsfonts}
\usepackage{amssymb}
\usepackage{amsxtra}
\usepackage{amsthm}
\usepackage{makecell}
\usepackage[skip=0pt, belowskip=-10pt]{caption}
\usepackage{subcaption}
\usepackage{float}
\usepackage{multirow}
\usepackage{booktabs}
%\usepackage{subfigure}
\usepackage{eso-pic}
%\usepackage{transparent}
\usepackage{graphicx}
\usepackage{tikz}
%\usepackage{longtable}
\usepackage{threeparttablex}

\newtheorem{df}{Definition}
\newtheorem{lm}{Lemma}
\newtheorem{prp}{Proposition}
\newtheorem{sprf}{Sketch of Proof}
\newtheorem{prf}{Proof}
\newtheorem{conjecture}{Conjecture}
\newtheorem{suffc}{Sufficient Condition}
\setbeameroption{hide notes}
\newcommand{\threelinebracer}{$\left. \begin{array}{c} \\ \\ \\ \end{array} \right\rbrace$}
\newcommand{\threelinebracel}{$\left. \begin{array}{c} \\ \\ \\ \end{array} \right\lbrace$}
\newcommand{\twolinebracer}{$\left. \begin{array}{c} \\ \\ \end{array} \right\rbrace$}
\newcommand{\twolinebracel}{$\left. \begin{array}{c} \\ \\ \end{array} \right\lbrace$}
\newcommand{\bd}{\partial}

\usepackage{pgf}  
%\logo{\pgfputat{\pgfxy(-1.2,-0.2)}{\pgfbox[center,base]{\includegraphics[height=12pt, keepaspectratio]{UA_Logo_Horizontal.eps}}} }

%\usebackgroundtemplate
%{
  %  \node[opacity=0.3, at=(current page.south east),anchor=south east,inner sep=0pt] 
    %\includegraphics[width=\paperwidth,height=20pt]{UA_Logo_Horizontal.eps}%
%}

\linespread{1}
\usepackage{parskip}
%\setlength{\itemsep}{1em} 
%\addtolength{\parskip}{5pt}
\DeclareMathSizes{12}{10}{8}{6}
%  \begin{itemize}}{\end{itemize}}
% Separate slides by \begin{frame} and \end{frame}.
\title[Willingness-to-pay for Warnings]{Willingness-to-pay for Warnings: Revision Discussion}
\author[A. Gaduh, P. McGee and A. Ugarov]{A. Gaduh, P. McGee and A. Ugarov}
\institute[]{}
\date{\today}
\newcommand{\sym}[1]{\ifmmode^{#1}\else\(^{#1}\)\fi}

\newcommand\BackgroundPic{%
\put(0,0){%
\parbox[b][\paperheight]{\paperwidth}{%
\vfill
\centering
%\includegraphics[width=\paperwidth,height=\paperheight,%keepaspectratio]{sancho.png}%
\vfill
}}}


\begin{document}

\begin{frame}
\frametitle{Sample(s) Structure}
\footnotesize
\begin{tabular}{l*{6}{c}}
\hline\hline
 & \multicolumn{2}{|c|}{All}  & \multicolumn{2}{c|}{$p \in\{0.1,0.3\}$} & \multicolumn{2}{|c|}{$p \in\{0.2,0.5\}$}\\
\hline
 & N & \% & N & \% & N & \%  \\
\hline
\multicolumn{7}{c}{All waves} \\
\hline
Male & 96 & 47 & 49 & 46 & 47 & 47 \\
Age$>$23yrs old & 16 & 8 & 8 & 7 & 8 & 8 \\
Students & 174 & 84 & 90 & 84 & 84 & 85 \\
Had statistics classes & 128 & 62 & 71 & 66 & 57 & 58 \\
\hline
\multicolumn{7}{c}{First waves} \\
\hline
Male & 43 & 21 & 22 & 21 & 21 & 21 \\
Age$>$23yrs old & 14 & 7 & 6 & 6 & 8 & 8 \\
Students & 88 & 43 & 46 & 43 & 42 & 42 \\
Had statistics classes & 63 & 31 & 37 & 35 & 26 & 26 \\
\hline
\multicolumn{7}{c}{Second wave} \\
\hline
Male & 53 & 26 & 27 & 25 & 26 & 26 \\
Age$>$23yrs old & 2 & 1 & 2 & 2 & 0 & 0 \\
Students & 86 & 42 & 44 & 41 & 42 & 42 \\
Had statistics classes & 65 & 32 & 34 & 32 & 31 & 31 \\
\hline
\end{tabular}

\end{frame}


\begin{frame}
\frametitle{Treatments}
\footnotesize
\begin{tabular}{l*{6}{c}}
\hline\hline
 & \multicolumn{3}{c} {Gremlins composition} & & \\
Prop. of black balls ($p$) & Honest & Black-eyed &White-eyed & FP rate &  FN rate\\
\hline
0.1, 0.2, 0.3, 0.5         &    2 & 0  &  0 & 0 & 0\\
0.1, 0.2, 0.3, 0.5         &    1 & 1  &  0 & 0.5 & 0  \\
0.1, 0.2, 0.3, 0.5         &    1 & 0  &  1 & 0 & 0.5  \\

0.1, 0.2, 0.3, 0.5         &    3 & 1  &  0 & 0.33 & 0  \\

0.1, 0.2, 0.3, 0.5          &    3 & 0  &  1 & 0 & 0.33 \\

0.1, 0.2, 0.3, 0.5           &    3 & 1  & 1 & 0.33 & 0.33\\

0.1, 0.2, 0.3, 0.5         &    5 & 1  & 0 & 0.2 & 0\\
0.1, 0.2, 0.3, 0.5         &    5 & 0  & 1 & 0 & 0.2    \\
0.1, 0.2, 0.3, 0.5         &    5 & 1 & 1  & 0.2 & 0.2  \\
 \multicolumn{6}{l} {New treatments} \\
0.1, 0.2, 0.3, 0.5         &    1 & 1  &  0 & 0.5 & 0  \\
0.1, 0.2, 0.3, 0.5         &    1 & 0  &  1 & 0 & 0.5  \\
0.1, 0.2, 0.3, 0.5         &    5 & 2  &  0 & 0.29 & 0  \\
0.1, 0.2, 0.3, 0.5         &   5 & 0  &  2 & 0 & 0.29  \\
0.1, 0.2, 0.3, 0.5         &   5 & 1  &  1 & 0.14 & 0.14  \\
\hline\hline
\end{tabular}



\end{frame}


\begin{frame}
\frametitle{CRT scores: just for the reference}
\begin{itemize}
\item Extended CRT scores are lower than I expected (2 out of 7 median), slightly higher for college graduates (4 median). But this beats some previous studies (Toplak et al, 2014) finding about 1.5 items answered correctly on average.
\end{itemize}
\begin{figure}[h]
  \includegraphics[scale=0.2]{Graphs/hist_crt.png}
\end{figure}

\end{frame}


%knuckleheadness measures, summary tables, BP, IP protection, then WTP table, WTP table with Tobit, WTP table with other stuff (GPA, CRT)



\begin{frame}
\frametitle{Blind and Informed Protection}
\footnotesize
\begin{itemize}
\item Tighter confidence intervals for blind protection (BP) as expected
\item More points in IP, narrower confidence intervals for existing points, still roughly correlates with BP
\end{itemize}
\begin{figure}[h]
  \includegraphics[scale=0.2]{Graphs/ip_response_comp.png}
\end{figure}
\end{frame}



\begin{frame}
\frametitle{Comment: intermediate priors with both conditions}
\footnotesize
\parbox{\textwidth}{

\textit{"For instance, on page 16,
Result 1 shows that both-error conditions have systematically lowest WTP. This pattern
might be suspicious since single-error conditions often produce extreme posteriors (0 or 1)
while both-error conditions tend to produce intermediate posteriors. The complexity level
is different. Likelihood insensitivity, rather than belief updating, might also explain the
valuation. In addition, almost all the both-error conditions generate very low WTPs, thus
the apparent overvaluation for them might “simply be due to reversion to the mean.”"}
}

\parbox{\textwidth}{
Response: We added new treatments with both error (technically one extra combination of gremlins but for different priors). The distrition of WTP for both errors doesn't concentrate near zero.
}


\end{frame}



\begin{frame}
\frametitle{Distribution of posteriors (both errors vs one error)}
\footnotesize
\begin{itemize}
\item The majority of uncertain cases has both errors and they are not concentrated near zeros/edges
\end{itemize}
\begin{figure}[h]
  \includegraphics[scale=0.2]{Graphs/posteriors_both.png}
\end{figure}

\end{frame}



\begin{frame}
\frametitle{Distribution of theoretical values}
\footnotesize
\begin{itemize}
\item New wave significanly beefs up treatments with intermediate value
\end{itemize}
\begin{figure}[h]
  \includegraphics[scale=0.25]{Graphs/hist_value_by_wave.png}
\end{figure}
\end{frame}


\begin{frame}
\frametitle{Distribution of theoretical values}
\footnotesize
\begin{itemize}
\item Both error conditions often result in significant WTP
\end{itemize}
\begin{figure}[h]
  \includegraphics[scale=0.25]{Graphs/hist_value_by_error.png}
\end{figure}
\end{frame}



\begin{frame}
\frametitle{Comments: pooling in summary tables}
\footnotesize
\parbox{\textwidth}{

\textit{1) More importantly, their
approach contradicts their own theory since they average responses across all subjects
and conditions, but their theory predicts that different types of people (risk-averse versus
risk-neutral) should show different patterns of FP/FN sensitivity. ”"}
}
\par
\textit{2) I hope the authors can revise Tables 2 and 3 accordingly since the pooling of priors and
FN and FP structures may be uninformative. Given that, it will be more straightforward to
check how the elicited posteriors, protection actions and WTPs change for different priors
and error types.}

\parbox{\textwidth}{
Response: Split it by prior too? Could be two tables. The danger is that readers would take a more detailed table too seriously, it's original goal was just to give very rough first impression. 
}


\end{frame}



\begin{frame}
\frametitle{Protection Summary}
\begin{itemize}

\item Similar: overprotection for white hints, underprotection for black hints with no FP; new - slight underprotection for FP$>$0,FP=0
\end{itemize}

\footnotesize

\begin{table}[H]\centering 

\label{tab:nonparIP}
\adjustbox{max width=\textwidth}{
	\begin{threeparttable}
	\begin{tabular}{cccccccc} \hline \hline
	\multirow{4}{6ex}{\centering \textbf{Row}}
			&\multicolumn{2}{c}{\centering \textbf{Signal Characteristics}} &\multirow{3}{12ex}{\centering \textbf{Hint}} 
			& \multirow{3}{10ex}{\centering \textbf{Posterior}} & \multirow{3}{10ex}{\centering \textbf{Share Protect}} 
			& \multirow{3}{10ex}{\centering \textbf{Share Optimal}} 
			& \multirow{3}{13ex}{\centering \textbf{$p$}} \\ %P-val $(H_0: (5)=(6))$
			\cmidrule(lr){2-3}
		& \multirow{2}{12ex}{\centering \textbf{False Positive}} 
			& \multirow{2}{12ex}{\centering \textbf{False Negative}} 
		\\
		\\
		&(1) & (2) & (3) & (4) & (5) & (6) & (7) \\
		\hline
			(1)&No&No&White&0.000&0.049&0.000&0.000\\
(2)&No&Yes&White&0.112&0.262&0.041&0.000\\
(3)&Yes&No&White&0.000&0.255&0.000&0.000\\
(4)&Yes&Yes&White&0.117&0.454&0.096&0.000\\
(5)&No&No&Black&1.000&0.824&1.000&0.000\\
(6)&No&Yes&Black&1.000&0.855&1.000&0.000\\
(7)&Yes&No&Black&0.520&0.810&0.869&0.043\\
(8)&Yes&Yes&Black&0.517&0.875&0.900&0.367\\

			\\[-1em]
		\hline\hline 
	\end{tabular} 
	\begin{tablenotes}[flushleft]
			\item\leavevmode\kern-\scriptspace\kern-\labelsep \footnotesize \textit{Notes: The p-value in column 7 is for the test of equality between the theoretical prediction (column 6) and the observed share of protection (column 5).} 
	\end{tablenotes}								
	\end{threeparttable}
	}
\end{table}
\end{frame}





\begin{frame}
\frametitle{Belief Errrors Summary}
\begin{itemize}
\item Overestimation for white hints, black hints with FP, underestimation if there are FN or no error. Very similar.
\end{itemize}

\footnotesize
\begin{table}[H]\centering 
\label{tab:nonparError}
\adjustbox{max width=\textwidth}{
	\begin{threeparttable}
	\begin{tabular}{ccccccc} 
	\hline \hline
		\multirow{4}{6ex}{\centering \textbf{Row}}
			&\multicolumn{2}{c}{\centering \textbf{Signal Characteristics}} &\multirow{3}{12ex}{\centering \textbf{Hint}} %P-val $(H_0: Error = 0)$
			& \multirow{3}{10ex}{\centering \textbf{Posterior}} 
			& \multirow{3}{12ex}{\centering \textbf{Updating Error$^*$}} & \multirow{3}{12ex}{\centering \textbf{$p$}}  \\ \cmidrule(lr){2-3}
		& \multirow{2}{10ex}{\centering \textbf{False Positive}} & \multirow{2}{12ex}{\centering \textbf{False Negative}} 
		\\
		\\
		& (1) & (2) & (3) & (4) & (5) & (6) \\
	
		\hline	
(1)&No&No&White&0.000&0.050&0.000\\
(2)&No&Yes&White&0.100&0.122&0.000\\
(3)&Yes&No&White&0.000&0.122&0.000\\
(4)&Yes&Yes&White&0.131&0.218&0.000\\
(5)&No&No&Black&1.000&-0.163&0.000\\
(6)&No&Yes&Black&1.000&-0.279&0.000\\
(7)&Yes&No&Black&0.550&0.039&0.130\\
(8)&Yes&Yes&Black&0.483&0.048&0.021\\

\\ 	[-1em]
\hline\hline
\end{tabular} 								
	\end{threeparttable}
	}
\end{table}
\end{frame}





\begin{frame}
\frametitle{WTP Summary}
\begin{itemize}
\item Overpaying: low priors - if there are FP, high priors - if there are FN; overpaying if both errors. 
\end{itemize}
\vspace{20pt}
\begin{table}[H]\centering \caption{Average WTP discrepancy (WTP-Value) by Signal Type} \begin{tabular}{ccccc} \hline \hline
\textbf{Prior$>$0.2}&\textbf{False-positive}&\textbf{False-negative}&\textbf{Mean WTP discrepancy}& \textbf{P($=0$)}\\ \hline
0&No&No&-0.135&0.465\\
0&No&Yes&-0.209&0.152\\
0&Yes&No&0.465&0.005\\
0&Yes&Yes&0.437&0.001\\
1&No&No&-0.077&0.698\\
1&No&Yes&0.496&0.013\\
1&Yes&No&-0.303&0.073\\
1&Yes&Yes&0.547&0.007\\
\hline \end{tabular} \end{table}

\end{frame}







\begin{frame}
\frametitle{Just IP responses regression for review}
\footnotesize
\begin{itemize}
\item Previous insights hold: FP/FN rates affect protection controlling on posteriors and beliefs
\end{itemize}

\begin{table}[htbp]\centering 
\caption{Informed Protection Response} 

\adjustbox{width=0.65\textwidth, keepaspectratio}{
	\begin{threeparttable}	
	\begin{tabular}{l*{4}{c}}
	\hline\hline
									&\multicolumn{1}{c}{(1)}&\multicolumn{1}{c}{(2)}&\multicolumn{1}{c}{(3)}&\multicolumn{1}{c}{(4)}\\
	\hline
		
		FP rate x (S=White)&    0.895\sym{***}&    0.943\sym{***}&    0.525\sym{***}&    0.571\sym{***}\\
                & (10.011)         & (10.145)         &  (5.518)         &  (5.850)         \\
FN rate x (S=White)&    0.537\sym{***}&    0.532\sym{***}&    0.307\sym{**} &    0.299\sym{**} \\
                &  (3.709)         &  (3.631)         &  (2.139)         &  (2.048)         \\
p$>$0.2         &    0.039\sym{**} &    0.041\sym{**} &    0.024         &    0.029         \\
                &  (2.408)         &  (1.965)         &  (1.558)         &  (1.457)         \\
S=Black         &    0.531\sym{***}&    0.542\sym{***}&    0.383\sym{***}&    0.374\sym{***}\\
                &  (5.161)         &  (4.758)         &  (3.653)         &  (3.268)         \\
FP rate x (S=Black)&   -0.032         &    0.025         &   -0.065         &   -0.000         \\
                & (-0.158)         &  (0.123)         & (-0.330)         & (-0.001)         \\
FN rate x (S=Black)&    0.103         &    0.069         &   -0.005         &   -0.021         \\
                &  (1.398)         &  (0.860)         & (-0.055)         & (-0.229)         \\
FP rate x (p$>$0.2)&                  &   -0.081         &                  &   -0.085         \\
                &                  & (-1.066)         &                  & (-1.081)         \\
FN rate x (p$>$0.2)&                  &    0.088         &                  &    0.048         \\
                &                  &  (0.891)         &                  &  (0.521)         \\
\hline
N               &     2424         &     2424         &     2424         &     2424         \\
Pseudo R-squared&    0.505         &    0.505         &    0.538         &    0.539         \\
Log-likelihood  & -830.188         & -829.168         & -773.731         & -773.018         \\


	
	\\ [-1em]
	\hline
	Subject FE & Yes & Yes & Yes & Yes \\
	Flexible controls for: \\
	\hspace{1.5ex} Posterior & Yes & Yes & Yes & Yes \\
	\hspace{1.5ex} Beliefs & No & No & Yes & Yes \\	
	\hline\hline
	\end{tabular}
							
	\end{threeparttable}
	}
\end{table}

\end{frame}




\begin{frame}
\frametitle{Comment: coefficients interpretation}
\footnotesize
\parbox{\textwidth}{

\textit{The authors write: “subjects tend to overvalue false-negative costs for low probability
events and overvalue false-positive costs for high probability events.” Where do we see
that in Table 5? The coefficients of FP costs, FN costs on column 4 and 5 are all positive.
Should it be “subjects tend to overvalue more false-positive costs (coeff:0.800 vs 0.204) for
low probability events and overvalue more false-negative (coeff: 0.407 vs 0.150) costs for
high probability events.”? When comparing coefficients, the authors should also report
results in statistical tests}
}

\parbox{\textwidth}{

The coefficients had changed, stil underreacting to FP for low priors, no real difference for high priors on average. Should be reframed+tests when needed.
}


\end{frame}


\begin{frame}
\frametitle{Main WTP regression}
\footnotesize
\begin{table}[htbp!]
\centering
\adjustbox{max width = \textwidth}{
	\begin{threeparttable}
	\caption{Deviations from Signal Value (WTP - Value) and Signal Characteristics}
	\label{tab:wtp_ols}
	\begin{tabular}{l*{5}{c}}
		\hline\hline
		&\multicolumn{3}{c}{\multirow{2}{*}{All}} & \multicolumn{2}{c}{Prior}\\ \cmidrule(lr){5-6}
		&&&& $\{.1,.2\}$ & $\{.3,.5\}$\\ 
		\cmidrule(lr){2-4} \cmidrule(lr){5-5} \cmidrule(lr){6-6}  
		&\multirow{1}{*}{(1)} & \multirow{1}{*}{(2)} & \multirow{1}{*}{(3)} & \multirow{1}{*}{(4)} & \multirow{1}{*}{(5)}\\
	\toprule
		FP costs       &       0.421   &       0.487   &       0.643   &       0.577   &       0.303   \\
               &     (0.081)***&     (0.126)***&     (0.158)***&     (0.180)***&     (0.308)   \\
FN costs       &       0.287   &       0.327   &       0.357   &       0.016   &       0.367   \\
               &     (0.046)***&     (0.084)***&     (0.088)***&     (0.216)   &     (0.085)***\\
Risk-averse $\times$ FP costs&               &      -0.329   &      -0.415   &      -0.243   &      -0.576   \\
               &               &     (0.225)   &     (0.257)   &     (0.285)   &     (0.427)   \\
Risk-averse $\times$ FN costs&               &      -0.355   &      -0.361   &      -0.352   &      -0.288   \\
               &               &     (0.124)***&     (0.135)***&     (0.292)   &     (0.128)** \\
Risk-loving $\times$ FP costs&               &       0.048   &       0.018   &       0.008   &       0.318   \\
               &               &     (0.179)   &     (0.213)   &     (0.290)   &     (0.408)   \\
Risk-loving $\times$ FN costs&               &       0.080   &       0.110   &       0.361   &       0.119   \\
               &               &     (0.107)   &     (0.117)   &     (0.341)   &     (0.118)   \\

Obs            &        1230   &        1230   &        1230   &         615   &         615   \\

		Subject FE & Yes & Yes & Yes & Yes & Yes \\
		Inaccurate Belief Interactions & No & No & Yes & Yes & Yes \\
		Prior Probability FE & No & No & No & Yes & Yes \\
		\hline\hline
	\end{tabular}
	\end{threeparttable}
}
\end{table}		

\end{frame}


\begin{frame}
\frametitle{Accounting for WTP bounds (Tobit), WTP as dependent variable, in theory sensitivity=-1}
\begin{table}[htbp!]
\centering
\adjustbox{max width = 0.5\textwidth}{
	\begin{threeparttable}
	\begin{tabular}{l*{5}{c}}
		\hline\hline
		&\multicolumn{3}{c}{\multirow{2}{*}{All}} & \multicolumn{2}{c}{Prior}\\ \cmidrule(lr){5-6}
		&&&& $\{.1,.2\}$ & $\{.3,.5\}$\\ 
		\cmidrule(lr){2-4} \cmidrule(lr){5-5} \cmidrule(lr){6-6}  
		&\multirow{1}{*}{(1)} & \multirow{1}{*}{(2)} & \multirow{1}{*}{(3)} & \multirow{1}{*}{(4)} & \multirow{1}{*}{(5)}\\
	\toprule
model          &               &               &               &               &               \\
FP costs       &      -0.804   &      -0.653   &      -0.262   &      -0.389   &       0.101   \\
               &     (0.121)***&     (0.012)***&     (0.016)***&     (0.015)***&     (0.019)***\\
FN costs       &      -0.321   &      -0.329   &      -0.211   &      -0.791   &       0.386   \\
               &     (0.061)***&     (0.007)***&     (0.009)***&     (0.017)***&     (0.006)***\\
Risk-averse $\times$ FP costs&               &      -0.346   &      -0.360   &      -0.069   &      -0.796   \\
               &               &     (0.013)***&     (0.023)***&     (0.021)***&     (0.027)***\\
Risk-averse $\times$ FN costs&               &      -0.342   &      -0.276   &      -0.307   &      -0.441   \\
               &               &     (0.008)***&     (0.013)***&     (0.028)***&     (0.009)***\\
Risk-loving $\times$ FP costs&               &       0.114   &       0.058   &       0.046   &       0.251   \\
               &               &     (0.012)***&     (0.017)***&     (0.015)***&     (0.025)***\\
Risk-loving $\times$ FN costs&               &       0.102   &       0.143   &       0.463   &       0.141   \\
               &               &     (0.008)***&     (0.010)***&     (0.020)***&     (0.008)***\\
Constant       &       2.233   &      -7.971   &     -13.035   &      -5.754   &      -9.871   \\
               &     (0.154)***&     (0.008)***&     (0.003)***&     (0.004)***&     (0.004)***\\
sigma          &               &               &               &               &               \\
Constant       &       1.990   &       1.302   &       1.270   &       0.994   &       0.835   \\
               &     (0.077)***&     (0.001)***&     (0.001)***&     (0.001)***&     (0.001)***\\
*.subject\_id   &          No   &         Yes   &         Yes   &         Yes   &         Yes   \\
Prob(FP=FN)    &       0.000   &       0.000   &       0.000   &       0.000   &       0.000   \\
Obs            &        1230   &        1230   &        1230   &         615   &         615   \\
[1em] Risk-Averse Subjects: \\ \hspace{0.5em} False Positive&               &      -1.999   &      -1.622   &      -1.458   &      -1.694   \\
\hspace{1em}  se&               &     (0.024)   &     (0.037)   &     (0.034)   &     (0.044)   \\
\hspace{1em} $ p$-value&               &     [0.000]   &     [0.000]   &     [0.000]   &     [0.000]   \\
[0.5em] \hspace{0.5em} False Negative&               &      -1.671   &      -1.487   &      -2.098   &      -1.055   \\
\hspace{1em}  se&               &     (0.014)   &     (0.020)   &     (0.042)   &     (0.014)   \\
\hspace{1em}  $ p$-value&               &     [0.000]   &     [0.000]   &     [0.000]   &     [0.000]   \\
\\[1em] Risk-Loving Subjects: \\ \hspace{0.5em} False Positive&               &      -1.538   &      -1.203   &      -1.343   &      -0.647   \\
\hspace{1em}  se&               &     (0.022)   &     (0.031)   &     (0.028)   &     (0.042)   \\
\hspace{1em}  $ p$-value&               &     [0.000]   &     [0.000]   &     [0.000]   &     [0.000]   \\
\\[0.5em] \hspace{0.5em} False Negative&               &      -1.227   &      -1.068   &      -1.328   &      -0.473   \\
\hspace{1em}  se&               &     (0.013)   &     (0.018)   &     (0.036)   &     (0.013)   \\
\hspace{1em}  $ p$-value&               &     [0.000]   &     [0.000]   &     [0.000]   &     [0.000]   \\

	\end{tabular}
	\end{threeparttable}
}
\end{table}	
\end{frame}






\begin{frame}
\frametitle{Cognitive determinants of WTP}
\begin{itemize}
\item No significant effects of CRT scores either on the level or on sensitivity of WTP
\end{itemize}
\footnotesize
\adjustbox{max width=.7\textwidth}{
{
\def\sym#1{\ifmmode^{#1}\else\(^{#1}\)\fi}
\begin{tabular}{l*{6}{c}}
\hline\hline
                &\multicolumn{1}{c}{(1)}&\multicolumn{1}{c}{(2)}&\multicolumn{1}{c}{(3)}&\multicolumn{1}{c}{(4)}&\multicolumn{1}{c}{(5)}&\multicolumn{1}{c}{(6)}\\
                &\multicolumn{1}{c}{}&\multicolumn{1}{c}{}&\multicolumn{1}{c}{}&\multicolumn{1}{c}{}&\multicolumn{1}{c}{}&\multicolumn{1}{c}{}\\
\hline
GPA$>$3.5=1     &     .131         &    -.451         &                  &                  &                  &                  \\
                &    (0.4)         &    (0.5)         &                  &                  &                  &                  \\
FP costs        &     .367\sym{*}  &     .352\sym{*}  &      .55\sym{***}&     .515\sym{***}&     .423\sym{***}&     .435\sym{***}\\
                &    (0.2)         &    (0.2)         &    (0.1)         &    (0.1)         &    (0.1)         &    (0.1)         \\
GPA$>$3.5=1 $\times$ FP costs&   -.0409         &   -.0279         &                  &                  &                  &                  \\
                &    (0.2)         &    (0.2)         &                  &                  &                  &                  \\
FN costs        &     .341\sym{***}&     .442\sym{***}&     .318\sym{***}&     .417\sym{***}&     .342\sym{***}&     .321\sym{***}\\
                &    (0.1)         &    (0.1)         &    (0.1)         &    (0.1)         &    (0.1)         &    (0.1)         \\
GPA$>$3.5=1 $\times$ FN costs&    -.158         &    -.228\sym{*}  &                  &                  &                  &                  \\
                &    (0.1)         &    (0.1)         &                  &                  &                  &                  \\
GPA$>$3.5       &        0         &        0         &                  &                  &                  &                  \\
                &      (.)         &      (.)         &                  &                  &                  &                  \\
$<$4 CRT errors=1&                  &                  &     .434         &   -.0602         &                  &                  \\
                &                  &                  &    (0.5)         &    (0.5)         &                  &                  \\
$<$4 CRT errors=1 $\times$ FP costs&                  &                  &    -.363         &    -.326         &                  &                  \\
                &                  &                  &    (0.2)         &    (0.2)         &                  &                  \\
$<$4 CRT errors=1 $\times$ FN costs&                  &                  &    -.134         &    -.225\sym{*}  &                  &                  \\
                &                  &                  &    (0.1)         &    (0.1)         &                  &                  \\
crt\_errors      &                  &                  &   .00699         &    .0212         &                  &                  \\
                &                  &                  &    (0.1)         &    (0.1)         &                  &                  \\
Stat. class     &                  &                  &                  &                  &     .218         &     .306         \\
                &                  &                  &                  &                  &    (0.2)         &    (0.3)         \\
Stat. class $\times$ FP costs&                  &                  &                  &                  &    -.122         &    -.152         \\
                &                  &                  &                  &                  &    (0.2)         &    (0.2)         \\
Stat. class $\times$ FN costs&                  &                  &                  &                  &   -.0727         &   -.0204         \\
                &                  &                  &                  &                  &    (0.1)         &    (0.1)         \\
Constant        &    -.345         &     .868\sym{**} &    -.556         &      .47         &    -.418\sym{**} &     .222         \\
                &    (0.3)         &    (0.4)         &    (0.7)         &    (0.7)         &    (0.2)         &    (0.2)         \\
Prior dummies   &       No         &      Yes         &       No         &      Yes         &       No         &      Yes         \\
\hline
Observations    &      492         &      492         &      606         &      606         &     1230         &     1230         \\
Adjusted \(R^{2}\)&     0.02         &     0.18         &     0.04         &     0.20         &     0.04         &     0.20         \\
\hline\hline
\multicolumn{7}{l}{\footnotesize Standard errors in parentheses}\\
\multicolumn{7}{l}{\footnotesize \sym{*} \(p<0.10\), \sym{**} \(p<0.05\), \sym{***} \(p<0.01\)}\\
\end{tabular}
}

}

\end{frame}

\begin{frame}
\frametitle{Demographic determinants of WTP}
\footnotesize
\begin{itemize}
\item Males and subjects with good quiz have slightly lower WTP. Sensitivities are higher only for subjects with the good  quiz (though it is endogenous obv)
\end{itemize}
\adjustbox{max width=0.9\textwidth}{
{
\def\sym#1{\ifmmode^{#1}\else\(^{#1}\)\fi}
\begin{tabular}{l*{7}{c}}
\hline\hline
                &\multicolumn{1}{c}{(1)}&\multicolumn{1}{c}{(2)}&\multicolumn{1}{c}{(3)}&\multicolumn{1}{c}{(4)}&\multicolumn{1}{c}{(5)}&\multicolumn{1}{c}{(6)}&\multicolumn{1}{c}{(7)}\\
                &\multicolumn{1}{c}{}&\multicolumn{1}{c}{}&\multicolumn{1}{c}{}&\multicolumn{1}{c}{}&\multicolumn{1}{c}{}&\multicolumn{1}{c}{}&\multicolumn{1}{c}{}\\
                &        b         &        b         &        b         &        b         &        b         &        b         &        b         \\
\hline
model           &                  &                  &                  &                  &                  &                  &                  \\
Male            &    -.388\sym{*}  &    -.341         &    -.567\sym{*}  &    -.381\sym{*}  &    -.401\sym{**} &     -.32         &    -.341         \\
Stat. class     &     .191         &     .433         &     .171         &     .145         &     .289         &     .466         &     .456         \\
ncorrect        &     .211\sym{***}&     .315\sym{***}&     .202\sym{***}&     .334\sym{***}&     .202\sym{***}&     .318\sym{***}&     .315\sym{***}\\
correctcrt      &                  &   -.0287         &                  &                  &                  &   -.0333         &   .00519         \\
gpa             &                  &     -.66         &                  &                  &                  &    -.423         &    -.572         \\
FP costs        &                  &                  &    -.859\sym{***}&    -.431\sym{***}&    -.633\sym{***}&    -.692\sym{***}&    -.581\sym{***}\\
FN costs        &                  &                  &    -.457\sym{***}&     -.28\sym{***}&    -.375\sym{***}&    -.279\sym{**} &    -.366\sym{***}\\
Male $\times$ FP costs&                  &                  &     .264         &                  &                  &                  &                  \\
Male $\times$ FN costs&                  &                  &    .0869         &                  &                  &                  &                  \\
Good quiz $\times$ FP costs&                  &                  &                  &    -.686\sym{***}&                  &                  &                  \\
Good quiz $\times$ FN costs&                  &                  &                  &    -.285\sym{***}&                  &                  &                  \\
Stat. class $\times$ FP costs&                  &                  &                  &                  &     -.17         &                  &                  \\
Stat. class $\times$ FN costs&                  &                  &                  &                  &   -.0752         &                  &                  \\
GPA$>$3.5=1 $\times$ FP costs&                  &                  &                  &                  &                  &   .00127         &                  \\
GPA$>$3.5=1 $\times$ FN costs&                  &                  &                  &                  &                  &     -.21         &                  \\
$<$4 CRT errors=1 $\times$ FP costs&                  &                  &                  &                  &                  &                  &    -.305         \\
$<$4 CRT errors=1 $\times$ FN costs&                  &                  &                  &                  &                  &                  &   -.0989         \\
Constant        &    -.348         &     1.37         &     .333         &    -.808         &     .178         &        1         &     1.46         \\
\hline
sigma           &                  &                  &                  &                  &                  &                  &                  \\
Constant        &     2.01\sym{***}&     2.05\sym{***}&     1.92\sym{***}&     1.91\sym{***}&     1.92\sym{***}&     1.96\sym{***}&     1.96\sym{***}\\
Prior dummies   &      Yes         &      Yes         &      Yes         &      Yes         &      Yes         &      Yes         &      Yes         \\
\hline
Observations    &     1230         &      492         &     1230         &     1230         &     1230         &      492         &      492         \\
Adjusted \(R^{2}\)&                  &                  &                  &                  &                  &                  &                  \\
\hline\hline
\end{tabular}
}

}

\end{frame}




\begin{frame}
\frametitle{Comment: coefficients interpretation}
\footnotesize
\parbox{\textwidth}{

\textit{"Similarly, given the importance of Figure 5, it would be nice if the authors could
include confidence interval of the regression coefficients, and present in more details
the regression specification."}
}

\parbox{\textwidth}{

See the graph with confidence intervals added. And also the same graph using Tobit to estimate sensitivities. Will add regression specification either into the figurenotes or into the text.
}


\end{frame}




\begin{frame}
\frametitle{WTP Sensitivity}
\begin{itemize}
\item We can never reject the hypotheses that empirical FP/FN sensitivities are the same, but somtimes cannot reject that they are equal to theoretical ones.
\end{itemize}
\begin{figure}[h]
\includegraphics[scale=0.3]{Graphs/sensit_comparison.png}
\end{figure}
\end{frame}



\begin{frame}
\frametitle{WTP Sensitivity (Tobit)}
\footnotesize
\begin{figure}[h]
\includegraphics[scale=0.3]{Graphs/sensit_comparison_tobit.png}
\end{figure}
\end{frame}








\begin{frame}
\frametitle{Comment: Cross-Task Consistency Checks}
\footnotesize
\parbox{\textwidth}{

\textit{1) However, the current version of the paper does not explore much of the relationship
between protection actions and WTP. Therefore, I encourage the authors to investigate
how the protection actions observed in the experiment affect WTP, which sets
the paper apart from the existing literature. The analysis would also lead to new
implications. Intuitively, WTP for signals is roughly affected by two factors: 1) the
understanding/preference (etc.) over information 2) anticipated protection action
taken by the subject’s future self. As shown in Table 7, subjects exhibit failure to 
distinguish FP and FN even when choosing their protection actions. Is the equal
sensitivity of WTP w.r.t. FP and FN driven by rational anticipation of the “bias”
in protective choice? Or Is it driven by the heuristic when subjects compute the
expected benefit of getting the signal? Depending on the answer, the result will
have different policy implications for encouraging the acquisition of warning signals. ”"}
}

\end{frame}




\begin{frame}
\frametitle{What Expected Bias Entails?}

\begin{itemize}
\item Easy to imagine different potential protection biases with different implications:
\begin{enumerate}
\item  I know that there are some signal structures for which I err in certain direction ("Always hide for that stupid tornado siren test on Wednesday"). Why not adjust the protection decision if knowing the bias direction?
\item  I know that I do not account for particular information when making the decision (fail to distinguish FP and FN rates), better theoretical grounds, easier to model.
\item I poorly understand particular signal structures (we already do observe that subjects with good quiz have higher WTP). Less assumptions to make and to use, but plausibly  belief errors can indicate understanding of signal structures.
\end{enumerate}
\item We imply that the second type of bias exists for WTP in which subjects do not differentiate FP and FN costs $\implies$ What if subjects are aware of that bias in IP too and pay less for signal structures known to suffer from this bias?
\end{itemize}
\end{frame}



\begin{frame}
\frametitle{Narrowing on Beliefs}
\begin{itemize}
\item I am against studying biases of the first type - too internally inconsistent.
\item Both types 2 and 3 imply poor use of information. Possible test: do they pay less for signals in which they have large BE errors?
\item We can write a simple model for type 2 bias by rewriting the signal purchase problem in terms of variables they do account for: $\pi$, proportion of dishonest gremlins.
\begin{itemize}
\item Still not clear how we test it? Bias correction significance is one option, but inconclusive because as a function of signal characteristics it can proxy for something else (preference non-linearity).
\end{itemize}
\item Note: these biases cannot explain paying for unused signals, and as average WTP roughly equals the risk-neutral value, either the bias has to be small, or risk preferences increase baseline WTP
\end{itemize}
\end{frame}




\begin{frame}
\frametitle{Do errors correlate with discounting WTP?}

\begin{itemize}
\item Belief error in round indeed strongly correlates with WTP (with subject and prior FE)
\end{itemize}
\adjustbox{max width=0.9\textwidth}{
\begin{tabular}{l*{3}{c}}
\hline\hline
                &\multicolumn{1}{c}{(1)}&\multicolumn{1}{c}{(2)}&\multicolumn{1}{c}{(3)}\\
                &\multicolumn{1}{c}{}&\multicolumn{1}{c}{}&\multicolumn{1}{c}{}\\
\hline
Subject-round-specific belief error&     .294         &    -.795\sym{***}&    -.723\sym{**} \\
                &    (0.3)         &    (0.3)         &    (0.3)         \\
FP costs        &                  &     .482\sym{***}&     .571\sym{***}\\
                &                  &    (0.1)         &    (0.1)         \\
FN costs        &                  &     .342\sym{***}&     .129         \\
                &                  &    (0.0)         &    (0.1)         \\
p$>$0.2 $\times$ FN costs&                  &                  &     .234\sym{***}\\
                &                  &                  &    (0.1)         \\
p$>$0.2 $\times$ FP costs&                  &                  &    -.321\sym{***}\\
                &                  &                  &    (0.1)         \\
Prior dummies   &      Yes         &      Yes         &      Yes         \\
\hline
Observations    &     1230         &     1230         &     1230         \\
Adjusted \(R^{2}\)&     0.50         &     0.56         &     0.56         \\
\hline\hline
\multicolumn{4}{l}{\footnotesize Standard errors in parentheses}\\
\multicolumn{4}{l}{\footnotesize \sym{*} \(p<0.10\), \sym{**} \(p<0.05\), \sym{***} \(p<0.01\)}\\
\end{tabular}

}
\end{frame}


\end{document}