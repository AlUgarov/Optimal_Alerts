\documentclass[11pt,hyperref={bookmarks=false}]{beamer}
\usetheme{Warsaw}
%\usetheme{Madrid}
%\usecolortheme{beaver}
\usefonttheme{professionalfonts}
% \usepackage[usenames,dvipsnames]{pstricks}
 %\usepackage{wallpaper}
 \usepackage{epsfig}
\definecolor{UniBlue}{RGB}{157,34,53}
\setbeamercolor{block title}{bg=UniBlue!70,fg=black}

\usepackage{psfrag,graphicx}
\usepackage{amsmath,amsfonts}
\usepackage{adjustbox}
\usepackage{lscape}
\usepackage{array,epsfig}
\usepackage{amsfonts}
\usepackage{amssymb}
\usepackage{amsxtra}
\usepackage{amsthm}
\usepackage{makecell}
\usepackage[skip=0pt, belowskip=-10pt]{caption}
\usepackage{subcaption}
\usepackage{float}
\usepackage{multirow}
\usepackage{booktabs}
%\usepackage{subfigure}
\usepackage{eso-pic}
%\usepackage{transparent}
\usepackage{graphicx}
\usepackage{tikz}
%\usepackage{longtable}
\usepackage{threeparttablex}

\newtheorem{df}{Definition}
\newtheorem{lm}{Lemma}
\newtheorem{prp}{Proposition}
\newtheorem{sprf}{Sketch of Proof}
\newtheorem{prf}{Proof}
\newtheorem{conjecture}{Conjecture}
\newtheorem{suffc}{Sufficient Condition}
\setbeameroption{hide notes}
\newcommand{\threelinebracer}{$\left. \begin{array}{c} \\ \\ \\ \end{array} \right\rbrace$}
\newcommand{\threelinebracel}{$\left. \begin{array}{c} \\ \\ \\ \end{array} \right\lbrace$}
\newcommand{\twolinebracer}{$\left. \begin{array}{c} \\ \\ \end{array} \right\rbrace$}
\newcommand{\twolinebracel}{$\left. \begin{array}{c} \\ \\ \end{array} \right\lbrace$}
\newcommand{\bd}{\partial}

\usepackage{pgf}  
%\logo{\pgfputat{\pgfxy(-1.2,-0.2)}{\pgfbox[center,base]{\includegraphics[height=12pt, keepaspectratio]{UA_Logo_Horizontal.eps}}} }

%\usebackgroundtemplate
%{
  %  \node[opacity=0.3, at=(current page.south east),anchor=south east,inner sep=0pt] 
    %\includegraphics[width=\paperwidth,height=20pt]{UA_Logo_Horizontal.eps}%
%}

\linespread{1}
\usepackage{parskip}
%\setlength{\itemsep}{1em} 
%\addtolength{\parskip}{5pt}
\DeclareMathSizes{12}{10}{8}{6}
%  \begin{itemize}}{\end{itemize}}
% Separate slides by \begin{frame} and \end{frame}.
\title[Willingness-to-pay for Warnings]{Willingness-to-pay for Warnings: Revision Discussion}
\author[A. Gaduh, P. McGee and A. Ugarov]{A. Gaduh, P. McGee and A. Ugarov}
\institute[]{}
\date{\today}
\newcommand{\sym}[1]{\ifmmode^{#1}\else\(^{#1}\)\fi}

\newcommand\BackgroundPic{%
\put(0,0){%
\parbox[b][\paperheight]{\paperwidth}{%
\vfill
\centering
%\includegraphics[width=\paperwidth,height=\paperheight,%keepaspectratio]{sancho.png}%
\vfill
}}}


\begin{document}
%\AddToShipoutPicture*{\BackgroundPic}

%\begin{frame}
%\titlepage
%\end{frame}

%%%%%%%%%%%%%%%%%%%%%%%%%%%%%%%%%%%%%%%%%%%%%%%%%%%%%%%%%%%%%%%%%%%%%%%%%%%%%%%%%%%%%%%%%%%%%%%%
%%%%%%%%%%%%%%%%%%%%%%%%%%%%%%%%%%%%%%%%%%%%%%%%%%%%%%%%%%%%%%%%%%%%%%%%%%%%%%%%%%%%%%%%%%%%%%%%


\begin{frame}
\frametitle{Sample(s) Structure}
\footnotesize
\begin{tabular}{l*{6}{c}}
\hline\hline
 & \multicolumn{2}{|c|}{All}  & \multicolumn{2}{c|}{$p \in\{0.1,0.3\}$} & \multicolumn{2}{|c|}{$p \in\{0.2,0.5\}$}\\
\hline
 & N & \% & N & \% & N & \%  \\
\hline
\multicolumn{7}{c}{All waves} \\
\hline
Male & 96 & 47 & 49 & 46 & 47 & 47 \\
Age$>$23yrs old & 16 & 8 & 8 & 7 & 8 & 8 \\
Students & 174 & 84 & 90 & 84 & 84 & 85 \\
Had statistics classes & 128 & 62 & 71 & 66 & 57 & 58 \\
\hline
\multicolumn{7}{c}{First waves} \\
\hline
Male & 43 & 21 & 22 & 21 & 21 & 21 \\
Age$>$23yrs old & 14 & 7 & 6 & 6 & 8 & 8 \\
Students & 88 & 43 & 46 & 43 & 42 & 42 \\
Had statistics classes & 63 & 31 & 37 & 35 & 26 & 26 \\
\hline
\multicolumn{7}{c}{Second wave} \\
\hline
Male & 53 & 26 & 27 & 25 & 26 & 26 \\
Age$>$23yrs old & 2 & 1 & 2 & 2 & 0 & 0 \\
Students & 86 & 42 & 44 & 41 & 42 & 42 \\
Had statistics classes & 65 & 32 & 34 & 32 & 31 & 31 \\
\hline
\end{tabular}

\end{frame}


%knuckleheadness measures, summary tables, BP, IP protection, then WTP table, WTP table with Tobit, WTP table with other stuff (GPA, CRT)


\begin{frame}
\frametitle{Blind Protection}
\footnotesize
\begin{figure}[h]
\includegraphics[scale=0.2]{Graphs/blind_prot_sta_all.png}
\end{figure}
\end{frame}







\begin{frame}
\frametitle{WTP Sensitivity}
\footnotesize

FP costs       &       0.421   &       0.487   &       0.643   &       0.577   &       0.303   \\
               &     (0.081)***&     (0.126)***&     (0.158)***&     (0.180)***&     (0.308)   \\
FN costs       &       0.287   &       0.327   &       0.357   &       0.016   &       0.367   \\
               &     (0.046)***&     (0.084)***&     (0.088)***&     (0.216)   &     (0.085)***\\
Risk-averse $\times$ FP costs&               &      -0.329   &      -0.415   &      -0.243   &      -0.576   \\
               &               &     (0.225)   &     (0.257)   &     (0.285)   &     (0.427)   \\
Risk-averse $\times$ FN costs&               &      -0.355   &      -0.361   &      -0.352   &      -0.288   \\
               &               &     (0.124)***&     (0.135)***&     (0.292)   &     (0.128)** \\
Risk-loving $\times$ FP costs&               &       0.048   &       0.018   &       0.008   &       0.318   \\
               &               &     (0.179)   &     (0.213)   &     (0.290)   &     (0.408)   \\
Risk-loving $\times$ FN costs&               &       0.080   &       0.110   &       0.361   &       0.119   \\
               &               &     (0.107)   &     (0.117)   &     (0.341)   &     (0.118)   \\

Obs            &        1230   &        1230   &        1230   &         615   &         615   \\

\end{frame}



\begin{frame}
\frametitle{Protection Summary}
\footnotesize

\begin{table}[H]\centering 
\caption{Average Protection by Signal Type} 
\label{tab:nonparIP}
\adjustbox{max width=\textwidth}{
	\begin{threeparttable}
	\begin{tabular}{cccccccc} \hline \hline
	\multirow{4}{6ex}{\centering \textbf{Row}}
			&\multicolumn{2}{c}{\centering \textbf{Signal Characteristics}} &\multirow{3}{12ex}{\centering \textbf{Hint}} 
			& \multirow{3}{10ex}{\centering \textbf{Posterior}} & \multirow{3}{10ex}{\centering \textbf{Share Protect}} 
			& \multirow{3}{10ex}{\centering \textbf{Share Optimal}} 
			& \multirow{3}{13ex}{\centering \textbf{$p$}} \\ %P-val $(H_0: (5)=(6))$
			\cmidrule(lr){2-3}
		& \multirow{2}{12ex}{\centering \textbf{False Positive}} 
			& \multirow{2}{12ex}{\centering \textbf{False Negative}} 
		\\
		\\
		&(1) & (2) & (3) & (4) & (5) & (6) & (7) \\
		\hline
			(1)&No&No&White&0.000&0.049&0.000&0.000\\
(2)&No&Yes&White&0.112&0.262&0.041&0.000\\
(3)&Yes&No&White&0.000&0.255&0.000&0.000\\
(4)&Yes&Yes&White&0.117&0.454&0.096&0.000\\
(5)&No&No&Black&1.000&0.824&1.000&0.000\\
(6)&No&Yes&Black&1.000&0.855&1.000&0.000\\
(7)&Yes&No&Black&0.520&0.810&0.869&0.043\\
(8)&Yes&Yes&Black&0.517&0.875&0.900&0.367\\

			\\[-1em]
		\hline\hline 
	\end{tabular} 
	\begin{tablenotes}[flushleft]
			\item\leavevmode\kern-\scriptspace\kern-\labelsep \footnotesize \textit{Notes: The p-value in column 7 is for the test of equality between the theoretical prediction (column 6) and the observed share of protection (column 5).} 
	\end{tablenotes}								
	\end{threeparttable}
	}
\end{table}
\end{frame}


\begin{frame}
\frametitle{Protection Summary}
\footnotesize

\begin{table}[H]\centering 
\caption{Average Protection by Signal Type} 
\label{tab:nonparIP}
\adjustbox{max width=\textwidth}{
	\begin{threeparttable}
	\begin{tabular}{cccccccc} \hline \hline
	\multirow{4}{6ex}{\centering \textbf{Row}}
			&\multicolumn{2}{c}{\centering \textbf{Signal Characteristics}} &\multirow{3}{12ex}{\centering \textbf{Hint}} 
			& \multirow{3}{10ex}{\centering \textbf{Posterior}} & \multirow{3}{10ex}{\centering \textbf{Share Protect}} 
			& \multirow{3}{10ex}{\centering \textbf{Share Optimal}} 
			& \multirow{3}{13ex}{\centering \textbf{$p$}} \\ %P-val $(H_0: (5)=(6))$
			\cmidrule(lr){2-3}
		& \multirow{2}{12ex}{\centering \textbf{False Positive}} 
			& \multirow{2}{12ex}{\centering \textbf{False Negative}} 
		\\
		\\
		&(1) & (2) & (3) & (4) & (5) & (6) & (7) \\
		\hline
			(1)&No&No&White&0.000&0.049&0.000&0.000\\
(2)&No&Yes&White&0.112&0.262&0.041&0.000\\
(3)&Yes&No&White&0.000&0.255&0.000&0.000\\
(4)&Yes&Yes&White&0.117&0.454&0.096&0.000\\
(5)&No&No&Black&1.000&0.824&1.000&0.000\\
(6)&No&Yes&Black&1.000&0.855&1.000&0.000\\
(7)&Yes&No&Black&0.520&0.810&0.869&0.043\\
(8)&Yes&Yes&Black&0.517&0.875&0.900&0.367\\

			\\[-1em]
		\hline\hline 
	\end{tabular} 
	\begin{tablenotes}[flushleft]
			\item\leavevmode\kern-\scriptspace\kern-\labelsep \footnotesize \textit{Notes: The p-value in column 7 is for the test of equality between the theoretical prediction (column 6) and the observed share of protection (column 5).} 
	\end{tablenotes}								
	\end{threeparttable}
	}
\end{table}
\end{frame}


\begin{frame}
\frametitle{Belief Errrors Summary}
\footnotesize
\begin{table}[H]\centering 
\caption{Average Updating Error by Signal Type} 
\label{tab:nonparError}
\adjustbox{max width=\textwidth}{
	\begin{threeparttable}
	\begin{tabular}{ccccccc} 
	\hline \hline
		\multirow{4}{6ex}{\centering \textbf{Row}}
			&\multicolumn{2}{c}{\centering \textbf{Signal Characteristics}} &\multirow{3}{12ex}{\centering \textbf{Hint}} %P-val $(H_0: Error = 0)$
			& \multirow{3}{10ex}{\centering \textbf{Posterior}} 
			& \multirow{3}{12ex}{\centering \textbf{Updating Error$^*$}} & \multirow{3}{12ex}{\centering \textbf{$p$}}  \\ \cmidrule(lr){2-3}
		& \multirow{2}{10ex}{\centering \textbf{False Positive}} & \multirow{2}{12ex}{\centering \textbf{False Negative}} 
		\\
		\\
		& (1) & (2) & (3) & (4) & (5) & (6) \\
	
		\hline	
(1)&No&No&White&0.000&0.050&0.000\\
(2)&No&Yes&White&0.100&0.122&0.000\\
(3)&Yes&No&White&0.000&0.122&0.000\\
(4)&Yes&Yes&White&0.131&0.218&0.000\\
(5)&No&No&Black&1.000&-0.163&0.000\\
(6)&No&Yes&Black&1.000&-0.279&0.000\\
(7)&Yes&No&Black&0.550&0.039&0.130\\
(8)&Yes&Yes&Black&0.483&0.048&0.021\\

\\ 	[-1em]
\hline\hline
\end{tabular} 								
	\end{threeparttable}
	}
\end{table}
\end{frame}




\begin{frame}
\frametitle{IP responses}
\footnotesize

\begin{table}[htbp]\centering 
\caption{Informed Protection Response} 

\adjustbox{width=0.7\textwidth, keepaspectratio}{
	\begin{threeparttable}	
	\begin{tabular}{l*{4}{c}}
	\hline\hline
									&\multicolumn{1}{c}{(1)}&\multicolumn{1}{c}{(2)}&\multicolumn{1}{c}{(3)}&\multicolumn{1}{c}{(4)}\\
	\hline
		
		FP rate x (S=White)&    0.895\sym{***}&    0.943\sym{***}&    0.525\sym{***}&    0.571\sym{***}\\
                & (10.011)         & (10.145)         &  (5.518)         &  (5.850)         \\
FN rate x (S=White)&    0.537\sym{***}&    0.532\sym{***}&    0.307\sym{**} &    0.299\sym{**} \\
                &  (3.709)         &  (3.631)         &  (2.139)         &  (2.048)         \\
p$>$0.2         &    0.039\sym{**} &    0.041\sym{**} &    0.024         &    0.029         \\
                &  (2.408)         &  (1.965)         &  (1.558)         &  (1.457)         \\
S=Black         &    0.531\sym{***}&    0.542\sym{***}&    0.383\sym{***}&    0.374\sym{***}\\
                &  (5.161)         &  (4.758)         &  (3.653)         &  (3.268)         \\
FP rate x (S=Black)&   -0.032         &    0.025         &   -0.065         &   -0.000         \\
                & (-0.158)         &  (0.123)         & (-0.330)         & (-0.001)         \\
FN rate x (S=Black)&    0.103         &    0.069         &   -0.005         &   -0.021         \\
                &  (1.398)         &  (0.860)         & (-0.055)         & (-0.229)         \\
FP rate x (p$>$0.2)&                  &   -0.081         &                  &   -0.085         \\
                &                  & (-1.066)         &                  & (-1.081)         \\
FN rate x (p$>$0.2)&                  &    0.088         &                  &    0.048         \\
                &                  &  (0.891)         &                  &  (0.521)         \\
\hline
N               &     2424         &     2424         &     2424         &     2424         \\
Pseudo R-squared&    0.505         &    0.505         &    0.538         &    0.539         \\
Log-likelihood  & -830.188         & -829.168         & -773.731         & -773.018         \\


	
	\\ [-1em]
	\hline
	Subject FE & Yes & Yes & Yes & Yes \\
	Flexible controls for: \\
	\hspace{1.5ex} Posterior & Yes & Yes & Yes & Yes \\
	\hspace{1.5ex} Beliefs & No & No & Yes & Yes \\	
	\hline\hline
	\end{tabular}
							
	\end{threeparttable}
	}
\end{table}

\end{frame}





\begin{frame}
\frametitle{WTP Sensitivity}
\footnotesize
\begin{figure}[h]
\includegraphics[scale=0.3]{Graphs/sensit_comparison.png}
\end{figure}
\end{frame}


\begin{frame}
\frametitle{WTP Sensitivity (Tobit)}
\footnotesize
\begin{figure}[h]
\includegraphics[scale=0.3]{Graphs/sensit_comparison_tobit.png}
\end{figure}
\end{frame}





		




\begin{frame}
\frametitle{}
\begin{itemize}
\item Comment: Cross-Task Consistency Checks

\end{itemize}
\end{frame}



\begin{frame}
\frametitle{Accounting for WTP bounds (Tobit)}
\begin{itemize}
\item Comment: Cross-Task Consistency Checks

\end{itemize}
\end{frame}




\begin{frame}
\frametitle{Figure 5 with confidence intervals (Tobit)}
\begin{itemize}
\item The authors write: “subjects tend to overvalue false-negative costs for low probability
events and overvalue false-positive costs for high probability events.” Where do we see
that in Table 5? The coefficients of FP costs, FN costs on column 4 and 5 are all positive.
Should it be “subjects tend to overvalue more false-positive costs (coeff:0.800 vs 0.204) for
low probability events and overvalue more false-negative (coeff: 0.407 vs 0.150) costs for
high probability events.”? When comparing coefficients, the authors should also report
results in statistical tests.

\end{itemize}


\end{frame}




\end{document}
