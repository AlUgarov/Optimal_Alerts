\documentclass[12pt,a4paper]{article}
\usepackage {amsmath,amsthm}
\usepackage{geometry}
\usepackage{titlesec}
 \geometry{a4paper, total={170mm,257mm}, left=20mm, right=20mm, top=25mm, bottom=25mm}
%\usepackage{tabularx}
\usepackage{sectsty}
\usepackage{natbib}
\bibliographystyle{econ}
\usepackage[hidelinks]{hyperref}
\usepackage{multirow}
\usepackage{makecell}
\usepackage{adjustbox}
\usepackage[normalem]{ulem}
\usepackage{threeparttable}
\usepackage{prettyref}

%\usepackage[colorlinks=true, urlcolor=Blue]{hyperref}
%\usepackage{titling}
\sectionfont{\fontsize{14}{15}\selectfont}
\subsectionfont{\fontsize{13}{15}\selectfont}

\titleformat*{\section}{\LARGE\bfseries}
\titleformat*{\subsection}{\Large\bfseries}


\usepackage{caption}
\captionsetup{belowskip=0pt,aboveskip=6pt}

\makeatletter
\setlength{\@fptop}{0pt}
\makeatother

\makeatletter
\newcommand*\exInput[1]{\@@input#1 }
\makeatother


\usepackage[dvipsnames]{xcolor}

\newcommand{\link}[1]{{\color{blue}\href{#1}{#1}}}

%TRACKING TOOL FOR ARYA: 
\newcommand{\agt}[1]{{\color{OliveGreen}#1}}
\newcommand{\agst}[1]{{\color{OliveGreen}\sout{#1}}}

%TRACKING TOOL FOR ALEX: 
\newcommand{\aut}[1]{{\color{Red}#1}}
\newcommand{\aust}[1]{{\color{Red}\sout{#1}}}

%TRACKING TOOL FOR Peter: 
\newcommand{\pmt}[1]{{\color{Blue}#1}}
\newcommand{\pmst}[1]{{\color{Blue}\sout{#1}}}


%\usepackage{cite}
\usepackage{booktabs}
\usepackage{float}
\usepackage{setspace}
\usepackage{placeins}
\usepackage[list=true]{subcaption}
\captionsetup[sub]{font=footnotesize}

\usepackage{graphicx}


\newtheorem{theorem}{Proposition}
\newtheorem{hypothesis}{Hypothesis}
	\def\hypothesisautorefname{Hypothesis}
\newtheorem{result}{Result}
	\def\resultautorefname{Result}
\def\sym#1{\ifmmode^{#1}\else\(^{#1}\)\fi}


\title{\Large Crying Wolf in the Lab\\}
\author{\large Arya Gaduh, Peter McGee, Alexander Ugarov*}
\begin{document}
\maketitle
\onehalfspacing
\begin{abstract}{Designing multiple kinds of signals involves trade-offs between false-positive and false-negative costs. We conduct a laboratory experiment to evaluate preferences over these trade-offs in a controlled environment. We find that the choices significantly diverge from the predictions of the model with a risk-neutral decision-maker as well as from some predictions of expected utility frameworks. Relative to a risk neutral decision-maker, willingness-to-pay overreacts to false-negative rates for low priors, but underreacts for high priors. Subjects' preferences demonstrate a reverse bias for false-positive rates. We find that this pattern is not consistent with the EU framework, but most consistent with a decision-making heuristic in which subjects do not differentiate between false-positive and false-negative rates when choosing signals.}


\vspace{10pt}
\begin{singlespace}

\noindent {\footnotesize{}Keywords: alarms, value of information, information economics, information design, medical tests, --- }{\thispagestyle{empty}}
\end{singlespace}
\end{abstract}

\vspace{180pt}
\newpage
\normalsize

\section{Introduction}

The 2010 gas blowout on Deep Horizon oil rig has killed 11 workers and caused one of the largest oil spills in history. The death toll was possibly aggravated by switching off a general safety alarm because its sirens interfered with workers' sleep.\footnote{https://www.nytimes.com/2010/07/24/us/24hearings.html} This illustrates the trade-off between false-positive and false-negative test results with false-positive rates leading to higher false alarm costs and false-negative resulting in missed events. 

Many real-life situations involve choosing binary tests to discover and prevent a negative outcome. Most binary tests transform continuous signals about the likelihood of an adverse state into simple yes/no prediction. This transformation relies on choosing a threshold for positive classification. Holding a continuous signal constant, a decrease in probability of no alarm in an adverse state (false-negative rate) corresponds to an increase in probability of alarm in a non-adverse state (false-positive rate). This trade-off motivates multiple discussions in medical diagnostics, alarm systems and extreme weather alerts. Despite ubiquity of binary alarms, there is little empirical evidence on how users evaluate alarms with different false-positive and false-negative rates. 

In order to understand preferences over these trade-offs, we study the demand for information in the framework with a potential protection action. The subject, first, receives a signal about the probability of an adverse event. Then she decides to protect or not. This environment describes several practically important scenarios including extreme weather alerts, medical testing and safety alarms. 

We find that the value of information in our setup weakly correlates with the willingness-to-pay. First, subjects on average underreact to quality of the signal, resulting in overpaying for low-quality signal and underpaying for high-quaity signals. Second, subjects tend to overreact to false-negative rates when the prior probability is low and overreact to false-positive positive rates when priors are high. We show that this pattern is most consistent with failure to estimate the effect of frequencies of false-positive and false-negative outcomes on the costs of using the signal. \citep{xu_revealed_2020} similarly finds that individuals do not properly account for priors and often choose tests not affecting optimal decisions even when more useful tests are available.

Our work is one of a few experimental studies measuring demand for information used for decision-making (instrumental information). Previous experimental studies studies the demand for signals in the prediction game in which subjects have to choose an optimal state under uncertainty. The field experiment conducted by \citep{hoffman_how_2016} finds that the demand for information increases with initial uncertainty, but decreases with the signal's accuracy. However, the decrease in accuracy is more modest than expected for a Bayesian decision-maker resulting in subjects underpaying for high-quality signals. The laboratory experiment of \citet{ambuehl_belief_2018} finds that subjects tend to underreact to the accuracy of the binary signal about state of the world, but put a premium on completely certain signals. The paper of \citet{xu_revealed_2020} also employs a prediction game setup to measure information preferences but varies priors on top of signal characteristics. She finds that many subjects choose non-instrumental over instrumental signals which is most consistent with failures of contingent reasoning on future value of information.

Our setup differs in two important aspects from \citep*{ambuehl_belief_2018, xu_revealed_2020}, because we study alerts and not prediction tasks. The subject faces a costly protection decision and not a prediction decision, resulting in three distinct payoffs: full payoff, full payoff minus protection costs and full payoff minus losses. It means that risk preferences affect the value of information and can change sensitivities to false-positive and false-negative rates. Our findings however are similar to prediction game findings. Consistent with \citet{ambuehl_belief_2018} we also find that subjects overvalue inaccurate signals, but we do not find a premium for certain signals.  And similar to \citet{xu_revealed_2020} we find that subjects commit reasoning errors leading to lower correlation between preferences and signal's usefulness in terms of cost reduction.

Due to its applicability for studying preferences over expectations, there is a larger stream of literature on the demand for non-instrumental information. \citet{eliaz_paying_2010} find that subjects are willing to pay for signals even when these signals are excessive for making optimal choices. Their design involves subjects choosing between two boxes with one box containing a prize of \$20. Most subjects pay just to know the probability of finding \$20 in box A even if this box is more likely to contain a prize in all the possible states. This finding is inconsistent with expected utility maximization but indicates instead having preferences for certainty before making choices.  Similar to this paper, \citet{masatlioglu_intrinsic_2017} also study preferences over information structures differing in false-positive and false-negative rates but their setup allows for a larger role of expectations. They find that for a positive potential outcome, most subjects prefer facing high false-negative rates rather than high false-positive rates. In other words, they tolerate uncertainty after negative signals better than uncertainty after positive signals. These preferences are salient: subjects require an average payment of 18-35 cents to switch to their least preferred information structure.

There is some mixed evidence that people update beliefs differently when these beliefs are ego-relevant or concern future gains and losses. \citet{eil_good_2011} find asymmetry in updating ego-relevant beliefs such as beauty and IQ. Subjects update more after receiving positive signals and do not update enough after negative signals. Additionally, subjects with high posterior ego-relevant beliefs are willing to pay to receive a more precise signals, but require a compensation for learning when their beliefs are low. In contrast, \citet{coutts_good_2019} does not find any updating asymmetry with respect to either ego-relevant beliefs or beliefs about future payoffs. 

Our paper is the first to measure value of information in the experimental setting of diagnostic tests or alarms. Previous work studies the use of alarms in context of medical testing, medical monitoring, safety alarms and extreme weather. Early literature on decision-making of medical professionals finds that doctors suffer from multiple biases when ordering testing, including  inaccurate posterior probability estimation due to availability heuristics, hindsight bias and regret \citep{bornstein_rationality_2001}. \citet{gigerenzer_helping_2007} find that most mammologists tend to overestimate the probability of cancer based on a positive result. Providing practitioners with natural frequencies instead of probabilities tends to reduce this bias.  

Patients' willingness-to-pay for medical tests is large and sensitive to test accuracy \citep{liang_acceptability_2003, howard_does_2009, neumann_willingness--pay_2012}. But test preferences also exhibit several abnormalities. First, users are willing to pay for tests having little or zero diagnostic value \citep{schwartz_enthusiasm_2004, neumann_willingness--pay_2012}. For example, \citet{schwartz_enthusiasm_2004} find that 73\% of Americans in their survey prefer a free full-body CT scan versus one thousand USD cash. However, medical professional do not recommend full-body CT scans for healthy people due to extreme likelihood of false-positive findings. Second, the framing of test accuracy seems to matter a lot. \citet{howard_does_2009} conduct a discrete-choice experiment to measure willingness-to-pay for the colorectal cancer screening. Their subjects agree to get 23 unnecessary colonoscopies in order to find one additional true cancer, but only 10.4 for reducing the number of cancers missed by one even though these descriptions are equivalent. Surprisingly, the perceived risk of cancer (prior) did not significantly affect the WTP in their study.

Our work also relates to the vast literature on demand for insurance and protection. Similar to our findings, several studies observe that the demand for insurance goes up after the recent experience with low probability events. Field evidence indicates that people under-insure with respect to rare natural disasters (Friedl et al, 2014). \citet{laury_insurance_2009} find no under-insurance for low-probability events in the laboratory setting. One offered explanation \citep{volkman-wise_representativeness_2015} is that subjects overweight recent evidence leading to under-insurance when there were no negative events in the recent past and to overinsurance after the fact. It is consistent with underweighting prior probabilities relative to more recent signals. 

The bias we are finding is similar to the base-rate and signal neglect phenomenons. Psychology researchers \citet{hammerton_case_1973} and \citet{kahneman_psychology_1973} first observed that subjects underweighted prior probabilities (base rates) when calculating posteriors. This phenomenon had received the name of \textit{base-rate neglect}. Multiple studies in economics then confirmed \citep*{grether_testing_1992, holt_update_2009} this phenomenon in incentivized laboratory experiments. Most of these studies find that subjects also underweight signals on top of priors.  We observe both phenomenons in responses to our belief elicitation task, but the calculation of signals' values differs substantially from the calculation of posterior probabilities. While the calculation of posterior probabilities would require using a Bayes formula, signal's value depends only on products of prior probabilities. However, we observe that subjects underestimate the effect of priors compared to theoretical predictions for an expected-utility decision-maker.


\section{Model}
\paragraph{Environment.} Consider an agent's purchase of threat assessment information. Let $\omega \in \{0,1\}$ denote the state of world, where 1 corresponds to an adverse event happening with probability $\pi$. Without a protective action, the agent has a lower utility in the adverse state. The protective action $a\in\{0,1\}$ employs a perfect technology: agents bear no losses when protected. Its cost is $c$. Preferences are described by the utility function which depends on the state of the world $\omega$, wealth $Y$, protective action $a$, and the potential loss in an adverse state $L > c$. Utility is separable in wealth, protection cost and the potential loss in the adverse state:\footnote{Separability condition does not impose additional restrictions on the utility function $U$ as long as the variation in wealth has limited range. More specifically, if $Y \in [Y_{min},Y_{max}]$ and $c<Y_{max}-Y_{min}, L<c+(Y_{max}-Y_{min})$, then the function $u(\cdot)$ can be constructed from segments of $U(\cdot,0,0), U(\cdot,1,0), U(\cdot,0,1)$.  While the resulting function $u(\cdot)$ is not necessarily monotonic, it is likely to be monotonic if protective actions and potential damages are relatively high.}
\begin{equation}
U=U(Y,a,\omega(1-a))=u(Y-ac-\omega(1-a)L)
\end{equation}

The agent can purchase a binary informative signal $s\in\{0,1\}$ about the state of the world before making a decision. Let $P_{ij}\equiv P(s=i|\omega=j)$ be the probability of a signal $s$ taking value $i$ conditional on the state of the world being $j$.  After receiving the signal, the agent updates her belief on the likelihood of the bad state to $\mu(s)$. Unless specified otherwise, we assume that she is Bayesian and her posterior belief equals:
\begin{equation}
\mu(s)= {\pi P_{s1} \over \pi P_{s1}+(1-\pi)P_{s0}}
\end{equation}

Without loss of generality, we also assume that a higher signal means a higher posterior probability of an adverse event $\mu(1)\geq\mu(0)$. Otherwise we can always re-label the signals.

\vspace{10pt}
\paragraph{Preferences.} If there is no signal, the agent protects if and only if it increases her expected utility:
\begin{equation}
EU_0=\max[u(Y-c),\pi u(Y-L)+(1-\pi) u(Y)]
\end{equation}
The signal can increase expected utility if the agent reacts differently to positive and negative signals. Under these assumptions, her expected utility with a signal is:
\begin{equation}
EU_s=\pi P_{11}u(Y-c)+\pi P_{01}u(Y-L)+(1-\pi)P_{10}u(Y-c)+(1-\pi)P_{00}u(Y)
\end{equation}

We consider the maximum amount $b$ which the agent is willing to pay for the signal. In our framework, $b$ is the price such that an agent is indifferent between paying $b$ to have the signal and not having a signal. Because the agent can always ignore a useless signal, the signal's value is bounded from below by zero. Hence it equals to the maximum between zero and the solution to the following equation:
\begin{equation}
\begin{split}
P(s=1)u(Y-b-c)+\pi P_{01}u(Y-b-L)+(1-\pi)P_{00}u(Y-b)=\\=\max[u(Y-c),\pi u(Y-L)+(1-\pi) u(Y)] 
\end{split}
\end{equation}

Here we use $P(s=1)\equiv \pi P_{11}+(1-\pi)P_{10}$ to denote the probability of a positive signal (alert). The left-hand side expression of this equation is a strictly decreasing function of $b$. Additionally, for $b\rightarrow \infty$ the left-hand side is smaller than the right-hand side. It implies that the equation (5) above has a at most one positive solution.

Obviously, perfectly accurate signals always have positive value $b>0$ because the payoff distribution with the signal first-order stochastically dominates the distribution without the signal. 
%If the agent protects without a signal, a perfect signal reduces the protection costs and if she takes chances, then it reduces losses in the adverse outcome from $L$ to $c<L$.  
However, it is harder to determine the value of the imperfect signal without imposing more restrictions on preferences as it requires weighing $u(Y-L)$ against $u(Y-c)$.


\paragraph{Risk-neutral agent.} If the agent is risk-neutral, the expression above collapses to:
$$b+P(s=1)c+\pi P_{01}L=\min[c,\pi L]$$

The signal's value is just:
\begin{equation}
b=\max[0,\min[c,\pi L]-P(s=1)c-\pi P_{01}L]
\end{equation}

We can express the WTP $b$ as a function of priors, false-positive, and false-negative rates. This is the equation we use in our empirical work:
\begin{equation}
b=\max[0,\min[c,\pi L]-\pi (1-P_{01})c-(1-\pi)P_{10}c-\pi P_{01}L]
\end{equation}\label{eq:rnWTP}

The sensitivity of (positive) value $b$ with respect to false-positive $P_{10}$ and false-negative $P_{01}$ rates is given by:
\begin{equation}
{db\over dP_{10}}=-(1-\pi)c
\end{equation}

\begin{equation}
{db\over dP_{01}}=-\pi(L-c)
\end{equation}
\vspace{10pt}

Both false-positive and false-negative rates decrease the (positive) signal's value. The effect is proportional to the adverse state probability for the false-negative rate and to the non-adverse state probability for the false-positive rates.

\paragraph{Risk Aversion Effects.} In a more general expected utility framework, risk aversion can both increase and decrease the signal's value. More specifically, risk aversion decreases the value when the protection costs are low: 

\begin{theorem}
 If protection costs are low $c<\pi L$, then a strictly risk-averse decision-maker pays less than a risk-neutral one.
\end{theorem} 
\begin{proof}
See the Appendix.
\end{proof}

%\normalsize
It is harder to make definite statements for lower risks or higher protection costs. For example, risk aversion increases value of a perfect signal as long as risk-averse decision-maker still chooses to not protect without a signal. This follows from the standard argument of increasing demand for insurance with risk aversion and the fact that the protection problem with a perfect signal is isomorphic to the insurance problem with deductible $c$. 
\aut{The following assumes that subjects do not use automatic blind protection? No, it just cancels out with differentiation by FP/FN rates.}
Next, we study the effect of false-positive and false-negative rates on the signal's value $b$. Assuming a differentiable utility function $u()$ we use implicit differentiation to derive sensitivities of WTP $b$ to false-positive and false-negative rates:

$${db\over dP_{10}}=-{(1-\pi)(u(Y-b)-u(Y-c-b))\over D(\pi, P_{01}, P_{10}, b)}$$
$${db\over dP_{01}}=-{\pi(u(Y-c-b)-u(Y-L-b))\over D(\pi, P_{01}, P_{10}, b)}$$

With the denominator equal to the expected marginal utility:
$$D(\pi, P_{01}, P_{10}, b)\equiv P(S=1)u'(Y-c-b)+\pi P_{01}u'(Y-L-b)+$$
$$+(1-\pi)P_{00}u'(Y-b)=E[MU]>0$$

It is clear that the signal's value decreases with false-positive and false-negative rates ${db\over dP_{10}},{db\over dP_{01}}<0$. We can also say a bit more about the sensitivity to false-negative rates:
\begin{theorem}
Risk-averse and imprudent decision-maker has higher sensitivity to false-negative rates as compared to a risk-neutral one.
\end{theorem}\label{thm:riskAverse}  
%\small
\begin{proof}
See the Appendix.
\end{proof}
%\normalsize

However, risk aversion can both increase and decrease subject's sensitivity to false-positive rates depending on the utility function curvature and signal's characteristics. Intuitively, an expected marginal utility of a strongly risk-averse subject with a bad signal can be lower than the average slope of the utility function between ($Y-c-b$) and ($Y-b$) which reduces sensitivity to false-positive rates. It can also be higher if either the signal is good or the curvature is small. We can only say that it is very likely that for low protection costs and small priors $\pi$ (leading to no automatic blind protection) the ratio of sensitivities to FP rates over FN rates should be lower for risk-averse subjects. 

\begin{theorem}
For low protection costs $c$ and small risks $\pi$, risk aversion lowers relative sensitivity to false-positive rates. 
\end{theorem}\label{thm:riskAverse2}  

\begin{proof}
See the Appendix.
\end{proof}
%\normalsize


%\vspace{10pt}
%\noindent The theory suggests two testable hypotheses. First, subjects' WTPs for a signal are equal to its value for risk-neutral agents. The model provides a simple benchmark to evaluate potential welfare gains from the signal. Depending on the assumptions on signal characteristics and risk preferences, agents can pay more or less than the risk-neutral benchmark.
%
%Second, subjects' preferences value the marginal costs from false-negative and false-positive information equally. The model of a risk-neutral agent suggests that they should. Our derivations above indicate that the relative weight of false-negative costs can be either below or above one depending on risk preferences only.




\vspace{20pt}

\section{Experimental Design}

Subjects received a USD 5 show-up fee and a USD 25 endowment that they might lose in the experiment. They must make a series of decisions in four sets of tasks: (i) Blind Protection; (ii) Informed Protection; (iii) Belief Elicitation; and (iv) Willingness to Pay Elicitation. Subjects took a quiz before each task and for every wrong response, the correct answer and explanation are given. Additionally, subjects had to answer extra questions if their responses to the Informed Protection questions were incorrect. Informed Protection is the first challenging task in the sequence and whose understanding is essential for the rest of the tasks.
Each task has 6 rounds, for a total of 24 rounds. One round is randomly selected as the payment round. A copy of the instruction is included in Appendix XX.

\bigskip
\noindent\textbf{Blind Protection (BP)}.\ \ \ Subjects must decide whether to insure (or “protect”) against an adverse event (a random draw of a black ball).  They are informed of the prior probability of a black ball. The cost to protect is USD 5. An unprotected subject who draws a black ball will lose USD 20. Subjects played six rounds with a probability (of a black ball) $p \in \{0.05,0.10,...,0.3\}$. In this task, subjects received no feedback on how each round would have been realized were it selected as the payment round.

\bigskip
\noindent\textbf{Informed Protection (IP)}.\ \ \ Similar to the BP task, subjects must make a protection decision given a prior probability of a black ball. However, in addition to the prior, subjects also received a signal with varying degrees of inaccuracy. Following \citet{coutts_good_2019}, we use a group of hinting gremlins to convey signal accuracy: a gremlin, randomly drawn from a group, gives out the signal. The gremlin is one of three types: (i) honest; (ii) ``black-swamp'' who always says that the ball is black; and (iii) ``white-swamp'' who always says that the ball is white. Figure XX illustrates how the different gremlin types were presented to the subjects. 

The composition of the group from which the gremlin is drawn determines signal accuracy: a higher share of black(white)-swamp gremlins will produce a signal with higher false-positive(negative) rate. Subjects know the group composition, but do not know which gremlin provides the hint. We vary the proportion of black balls in the box (prior probability of a black ball) and the composition of gremlins (signal quality) between rounds.  

\bigskip
\noindent\textbf{Belief Elicitation (BE)}.\ \ \ 
%We use the BE task to elicit subjects' beliefs about the likelihood of an adverse event and an adverse signal conditional on prior and signal characteristics in an incentive-compatible way. Similar to 
As in the IP task, subjects were told the prior probability of a black ball and the hinting gremlin's group composition. However, instead of making a protection decision, subjects are asked to estimate the probability that: (i) the ball is black ball when the gremlin says that it is white; (ii) the ball is black when the gremlin says that it is black. 

To elicit incentive-compatible responses, we follow the stochastic version of the Becker-DeGroot-Marshak mechanism developed by \citet{grether_testing_1992} and \citet{holt_update_2009}: the subject submits their belief of the probability of the event $\mu \in [0,1]$. If this belief is above some uniform random number $r\in[0,1]$, they receive the payoff $x$ only if the stated event happens. Otherwise their payoff is determined by an independent lottery which pays $x$ with probability $r$ and 0 otherwise.\footnote{The benefit of this mechanism versus other probability elicitation mechanism (for example, quadratic scoring) is that reporting truthfully is a dominant strategy regardless of risk preferences \citep{karni_mechanism_2009-1}. The only requirements a subject needs to satisfy are probabilistic sophistication and dominance: they rank lotteries based on their probabilities only and prefer higher probabilities of higher payoffs.} We also provide our subjects with the heuristics that under this mechanism, truthful reporting of beliefs is the dominant strategy.

\bigskip
\noindent\textbf{Willingness to Pay Elicitation (WTPE)}.\ \ \ The WTPE task measures subjects' willingness to pay (WTP) for signals. Subjects know the prior probability of a black ball and the group composition of the gremlins that will determine signal quality.  We then ask subjects for their WTP for a hint. Subjects can choose a value from USD 0 to 5 with USD 50-cent increments. Their decisions are incentive compatible: if a WTPE round is selected as the payment round, a random price of a hint will be drawn. If that price exceeded the subjects' WTP, they will play a BP round. Otherwise, the subject would pay their WTP and play an IP round.  

\vspace{10pt} 

After completing the WTPE task, subjects were asked a few demographic questions. The session concluded with the random selection and realization of the payment round, after which subjects were paid and dismissed. Table~\ref{tab:treatments} shows the values of the different priors in our treatments, as well as the gremlin groupings (along with the associated false positive and false positive rates) that we used for the different tasks.

%%%%%%%%%%%%%%%%%%%%%%%%%%%%%%%%%%%%%%%%%%%%%%%%%%%%%%%%%%%%%%%%%%%%%%%%%%%%%%%%%%%%%%%%%%%%%%%%%%%%%%%%
\begin{table}[h!]
\caption{List of Treatments} \label{tab:treatments}
\begin{tabular}{l*{6}{c}}
\hline\hline
 & \multicolumn{3}{c} {Gremlins composition} & & \\
Prop. of black balls ($p$) & Honest & Black-eyed &White-eyed & FP rate &  FN rate\\
\hline
0.1,0.2,0.3,0.5         &    2 & 0  &  0 & 0 & 0\\

0.1,0.2,0.3,0.5         &    3 & 1  &  0 & 0.333 & 0  \\

0.1,0.2,0.3,0.5          &    3 & 0  &  1 & 0 & 0.333 \\

0.1,0.2,0.3,0.5           &    3 & 1  & 1  0& 0.333 & 0.333\\

0.1,0.2,0.3,0.5         &    5 & 1  & 0 & 0.2 & 0\\
0.1,0.2,0.3,0.5         &    5 & 0  & 1 & 0 & 0.2    \\
0.1,0.2,0.3,0.5         &    5 & 1 & 1  & 0.2 & 0.2  \\

\end{tabular}


\end{table}

%%%%%%%%%%%%%%%%%%%%%%%%%%%%%%%%%%%%%%%%%%%%%%%%%%%%%%%%%%%%%%%%%%%%%%%%%%%%%%%%%%%%%%%%%%%%%%%%%%%%%%%%

We conducted this experiment in the Behavioral Business Research Lab (BBRL) at the University of Arkansas between October and November 2021.  The experiment was implemented using Qualtrics. There were a total of 105 subjects. 84 percent of the subjects were university students and 41 percent were male.  About 60 percent of the subjects had taken at least one statistics course. On average, including the show-up fee, subjects received around USD 26 for a session lasting around 45 minutes.  


\vspace{20pt}
\section{Subject Decisions By Task}\label{sec:sanity}

We use the BP, IP, and BE tasks to construct measures of the components of WTP suggested by our model. The BP task reveals subjects' responses to the prior and their risk preference. The IP task provides insights into how signal characteristics affect their protection decisions. Finally, the BE task shows how subjects would update for a given signal quality. Some of these tasks can be quite challenging. However, we show below that on average, subjects appear to exhibit a reasonable understanding of these tasks.

\subsection{Blind Protection}

Figure~\ref{fig:ProtResponse} plots the probability of protection decision against posterior probability of a black ball for the BP task, where the posterior is equivalent to the prior, and the IP task. On aggregate, subjects protect more with a higher probability of a negative outcome: only 13\% subjects protect when the probability of a black ball is 10\% in contrast to 70\% protecting when the probability is 30\%. 

At the individual level, BP responses indicate significant heterogeneity in risk preferences. For approximately 70\% of subjects (\agt{XXX X/Y XXX}), protection action increases monotonically in probability. The remaining 30\% make at least one switch from protecting to not protecting and back, which is inconsistent with EU maximization. Among these switchers, however, 83\% (24/39) skip only a single increment of the presented probability scale, suggesting an inattention error.\footnote{For comparison, Holt and Laury (2002) for a similar instrument find that 28 of 212 subjects (13\%) switched back to a low-risk option with an increasing likelihood of high payoffs in a risky option at least once.} 

Risk-neutral agents who maximize their expected utility should start protecting when the prior exceeds 0.25 (the ratio of the protection cost to the potential loss = \$5/\$20). Many of our subjects started protecting at lower priors, indicating strict risk aversion.\footnote{As a reference, switching at the probability 0.1 corresponds to a CRRA risk aversion $\theta=2$, while switching at 0.2 corresponds to $\theta=0.57$.}  A smaller group of subjects makes choices consistent with risk loving by protecting at a probability of 0.3 or never protecting. 

\subsection{Informed Protection}

In the IP task, subjects receive a signal (on top of a prior) to help them form a posterior probability of a black ball. 
Figure~\ref{fig:ProtResponse} shows that protection actions are increasing in the posteriors. Roughly 28\% of subjects break monotonicity in their protection responses with respect to posterior probabilities --- approximately the percentage of non-monotonic responses in the BP task.\footnote{ They do not protect for some treatments with posterior probability $P$ while protecting for a posterior probability $P'<P$.}  At the individual level, we also find that the total number of times subjects protect in the BP task significantly correlates with their likelihood to protect in the IP task conditional on posteriors, but this explains only a very small part ($<$1\%) of variation in the IP decisions.\footnote{We use LPM to estimate this relationship, and while the coefficient on the total number of protection choices is significant at 99\%, $R^2$ increases from 0.295 to 0.3.} 


%%%%%%%%%%%%%%%%%%%%%%%%%%%%%%%%%%%%%%%%%%%%%%%%%%%%%%%%%%%%%%%%%%%%%%%%%%%%%%%%%%%%%%%%%%%%%%%%%%%%%%%%%%%%%%%%%%%

\begin{figure}[H]
\centering
\caption{Average Protection Response} \label{fig:ProtResponse}
\begin{subfigure}[t]{.45\textwidth}
  \centering
\includegraphics[width=\textwidth]{Graphs/blind_prot_sta.png}
\caption{Note BP. Put note in latex instead of in figure title.}
\end{subfigure}
\begin{subfigure}[t]{.45\textwidth}
  \includegraphics[width=\textwidth]{Graphs/ip_response.png}
\caption{Note IP. Put note in latex instead of in figure title.}
\end{subfigure}
%\begin{subfigure}[t]{.48\textwidth}
  %\centering
  %\includegraphics[width=\textwidth]{Graphs/ip_response_lpoly.png}
%
%\end{subfigure}
\end{figure}
%\pmt{Eventually we need to add notes to the figures}
%%%%%%%%%%%%%%%%%%%%%%%%%%%%%%%%%%%%%%%%%%%%%%%%%%%%%%%%%%%%%%%%%%%%%%%%%%%%%%%%%%%%%%%%%%%%%%%%%%%%%%%%%%%%%%%%%%%

Table~\ref{tab:nonparIP} presents the average protection decisions by signal characteristics. The first three columns summarize the signal characteristics information that is available to the subject. Column 4 shows the posterior probability of a black ball averaged across all the treatments within a group. Column 5 shows the subjects' share of empirical protection responses, next to the theoretical optimum for risk-neutral subjects in Column 6. Column 7 presents the $p$-value for a test of equality between empirical and theoretical protection responses.

We have three main findings. First, we find that regardless of FP and FN rates, a black signal substantially increases the share of protection decisions.  Second, subjects' protection decisions in most treatments significantly deviate from what is optimal for risk-neutral subjects, evidenced by column 7. Subjects tend to overprotect when facing white signals (rows 1--4) and underprotect when facing black signals (rows 5--8). The exceptions are treatments with a black signal and positive FP rates where the statistical equality of columns 5 and 6 cannot be rejected (rows 7--8).

%\textbf{In light of BP decisions it is not surprising that subjects do not behave as risk-neutral agents, but some biases cannot be explained by the expected utility maximization for any degree of risk aversion.} 
Finally, we find that some deviations cannot be explained by the expected utility maximization for any degree of risk aversion. For example, consider rows 1 and 3: even though an increase in the signal's FP rate does not change the posterior (because the signal is white), the protection rate increases by 6 percentage points (pp.). Similarly, row 4 shows that when both FP and FN are positive, the protection rate increases to 56 percent --- even though the average (maximum) posterior probability for the signal characteristics is just 13 percent. As a benchmark, with no signal in the BP task, only 13~(32) percent of subjects chose to protect when the probability is 10~(15) percent. 

%%%%%%%%%%%%%%%%%%%%%%%%%%%%%%%%%%%%%%%%%%%%%%%%%%%%%%%%%%%%%%%%%%%%%%%%%%%%%%%%%%%%%%%%%%%%%

\begin{table}[H]\centering 
\caption{Average Protection by Signal Type} 
\label{tab:nonparIP}
\adjustbox{max width=\textwidth}{
	\begin{threeparttable}
	\begin{tabular}{cccccccc} \hline \hline
	\multirow{4}{6ex}{\centering \textbf{Row}}
			&\multicolumn{3}{c}{\centering \textbf{Signal Characteristics}} 
			& \multirow{3}{10ex}{\centering \textbf{Posterior}} & \multirow{3}{10ex}{\centering \textbf{Share Protect}} 
			& \multirow{3}{10ex}{\centering \textbf{Share Optimal}} 
			& \multirow{3}{13ex}{\centering \textbf{P-val $(H_0: (5)=(6))$}} \\ 
			\cmidrule(lr){2-4}
		&\multirow{2}{12ex}{\centering \textbf{Signal}} & \multirow{2}{12ex}{\centering \textbf{False Positive}} 
			& \multirow{2}{12ex}{\centering \textbf{False Negative}} 
		\\
		\\
		&(1) & (2) & (3) & (4) & (5) & (6) & (7) \\
		\hline
			(1)&No&No&White&0.000&0.049&0.000&0.000\\
(2)&No&Yes&White&0.112&0.262&0.041&0.000\\
(3)&Yes&No&White&0.000&0.255&0.000&0.000\\
(4)&Yes&Yes&White&0.117&0.454&0.096&0.000\\
(5)&No&No&Black&1.000&0.824&1.000&0.000\\
(6)&No&Yes&Black&1.000&0.855&1.000&0.000\\
(7)&Yes&No&Black&0.520&0.810&0.869&0.043\\
(8)&Yes&Yes&Black&0.517&0.875&0.900&0.367\\

			\\[-1em]
		\hline\hline 
	\end{tabular} 
	\begin{tablenotes}[flushleft]
			\item\leavevmode\kern-\scriptspace\kern-\labelsep \footnotesize \textit{Notes:} 
	\end{tablenotes}								
	\end{threeparttable}
	}
\end{table}



%\bigskip\noindent\textbf{Belief Elicitation.}\ \ \ 
\subsection{Belief Elicitation}
Subject decisions in the IP task capture how signals are used to make protection decisions, but conflate risk preferences with potential errors in updating posteriors.  The BP task can be used to construct a measure of the former. Here, we use the BE task to measure the latter.  


%%%%%%%%%%%%%%%%%%%%%%%%%%%%%%%%%%%%%%%%%%%%%%%%%%%%%%%%%%%%%%%%%%%%%%%%%%%%%%%%%%%%%%%%%%%%%%%%%%%%%%%%%%%%%%%%%%%

\begin{figure}[H]
	\centering
	\caption{Errors in Bayesian Updating} \label{fig:BeliefUpdate}
	\subcaptionbox{Error Distribution}{\includegraphics[width=.48\textwidth]{Graphs/hist_belief_error_s3.png}}
	\hfill
	\subcaptionbox{Error v. Posterior}{\includegraphics[width=.48\textwidth]{Graphs/updating_s3.png}}
	\hfill
	\vspace{2em}
	\subcaptionbox{Error Distribution, Certain Posterior}{\includegraphics[width=.48\textwidth]{Graphs/hist_belief_error_s5.png}}
	\hfill
	\subcaptionbox{Error v. Posterior, Certain Posterior}{\includegraphics[width=.48\textwidth]{Graphs/updating_s5.png}}
	\hfill
	\vspace{2em}
	\subcaptionbox{Error Distribution, Uncertain Posterior}{\includegraphics[width=.48\textwidth]{Graphs/hist_belief_error_s4.png}}
	\hfill
	\subcaptionbox{Error v. Posterior, Uncertain Posterior}{\includegraphics[width=.48\textwidth]{Graphs/updating_s4.png}}
	\hfill
\end{figure}
%%%%%%%%%%%%%%%%%%%%%%%%%%%%%%%%%%%%%%%%%%%%%%%%%%%%%%%%%%%%%%%%%%%%%%%%%%%%%%%%%%%%%%%%%%%%%%%%%%%%%%%%%%%%%%%%%%%

We define updating errors as the difference between the posterior and subjects' elicited belief on the posterior probability of a black ball for a given signal.  The left-hand column of Figure~\ref{fig:BeliefUpdate} shows the distribution of the updating errors, while its right-hand column presents a scatter plot of the elicited beliefs against the true posterior with a fitted line.
%Since subjects were only given signal characteristics and not true posteriors, their IP responses reflect, inter alia, their ability to infer the true posteriors from signal characteristics. (I'm not sure I like my sentence better)
Panel A uses all observations and suggests that, while errors occur, beliefs are still sensible. The distribution of updating errors is centered at 0, with roughly one-half (51\%) concentrated within +/- 0.1 interval around zero. Overall, the correlation between the elicited beliefs and the true posteriors was 0.653.  

Some combinations of priors and signals correspond to completely certain posteriors where updating should be trivial.  
%Panel B includes only the 44\% beliefs elicited for an uncertain posterior. 
Panel~B plots such cases, which account for 56\% of the sample and includes: (i) treatments with all-honest gremlins; and (ii) treatments with obviously irrelevant dishonest gremlins (e.g., a gremlin from a group with honest and white-swamp gremlins announcing that the ball is black --- or vice versa). Reassuringly, 69\% of reported beliefs are correct. About half of the errors involve reporting a probability of one when it should have been zero.

Meanwhile, Panel~C plots observations without such these observations. The median error is now -0.12, with with 90\% of errors between -0.48 and 0.3, suggesting that subjects tend to overestimate the likelihood of adverse events for uncertain posteriors. The correlation between beliefs and posteriors in this sub-sample falls to 0.571.  

The overall pattern of belief updating is consistent with the existing literature which shows that despite updating in the correct direction, people tend to underreact both to priors and the signals. The effect of underweighting priors --- first noted in the psychology literature \citep*{phillips_conservatism_1966-1, tversky_belief_1971, kahneman_subjective_1972} --- is known as \emph{representativeness bias} or \emph{base-rate neglect}. Using the regression approach of \citet{grether_bayes_1980}, we find both base-rate neglect and signal underweighting. Our estimates of these parameters are significantly below one with $\hat \alpha=0.43$ $\hat \beta=0.25$ (see Column 1 in \ref{belief_decomposition}). These values are within the range found by the meta-analysis of \citet{benjamin_chapter_2019} which calculates the average $\hat \alpha$ estimate to be around 0.22 (0.4 for incentivized studies only) and the average $\hat \beta$ to be 0.6 (0.43 for incentivized) for studies (like ours) that presented their signals simultaneously.\footnote{The common name for this kinds of experiments is \emph{bookbag-and-poker-chip experiments}} 

%%%%%%%%%%%%%%%%%%%%%%%%%%%%%%%%%%%%%%%%%%%%%%%%%%%%%%%%%%%%%%%%%%%%%%%%%%%%%%%%%%%%%%%%%%%%%

\begin{table}[H]\centering 
\caption{Average Updating Error by Signal Type} 
\label{tab:nonparError}
\adjustbox{max width=\textwidth}{
	\begin{threeparttable}
	\begin{tabular}{ccccccc} 
	\hline \hline
 \multirow{4}{6ex}{\centering \textbf{Row}}& \multicolumn{3}{c}{\centering \textbf{Signal Characteristics}}
			& \multirow{3}{10ex}{\centering \textbf{Posterior}}  
			& \multirow{3}{12ex}{\centering \textbf{Updating Error$^*$}} & \multirow{3}{12ex}{\centering \textbf{P-val $(H_0: Error = 0)$}}  \\ \cmidrule(lr){2-4}
		& \multirow{2}{10ex}{\centering \textbf{False Positive}} & \multirow{2}{12ex}{\centering \textbf{False Negative}} 
			& \multirow{2}{10ex}{\centering \textbf{Signal}} 
		\\
		\\
		& (1) & (2) & (3) & (4) & (5) \\
	
		\hline	
(1)&No&No&White&0.000&0.050&0.000\\
(2)&No&Yes&White&0.100&0.122&0.000\\
(3)&Yes&No&White&0.000&0.122&0.000\\
(4)&Yes&Yes&White&0.131&0.218&0.000\\
(5)&No&No&Black&1.000&-0.163&0.000\\
(6)&No&Yes&Black&1.000&-0.279&0.000\\
(7)&Yes&No&Black&0.550&0.039&0.130\\
(8)&Yes&Yes&Black&0.483&0.048&0.021\\

\\ 	[-1em]
\hline\hline
\end{tabular} 
	\begin{tablenotes}[flushleft]
			\item\leavevmode\kern-\scriptspace\kern-\labelsep \footnotesize \textit{Notes:} $^* \text{Updating error} = Posterior - Belief$. 
	\end{tablenotes}								
	\end{threeparttable}
	}
\end{table}
%%%%%%%%%%%%%%%%%%%%%%%%%%%%%%%%%%%%%%%%%%%%%%%%%%%%%%%%%%%%%%%%%%%%%%%%%%%%%%%%%%%%%%%%%%%%%

Table~\ref{tab:nonparError} summarizes how the updating errors vary with signal characteristics. We find that subjects overestimate the probability of a black ball when they received a white signal. Introducing FP rates to the signal exacerbated their upward bias. To illustrate, consider the change between rows 1 and 3, where introducing a FP rate would not change the posterior because the signal is white. Yet, subjects update their posterior upward, magnifying their updating error. The FN rates also have a similar effect of exacerbating this upward bias for a white signal.

The updating bias for black signals, however, varies by the information structure. Subjects slightly underestimate the probability even when the signal is honest, but introducing FN rates leads subjects to underestimate it further. To illustrate, the introduction of a FN rate given a black signal does not change the posterior (rows 5 and 6), but subjects decrease their beliefs. When there is a risk of a false-positive (i.e., FP$>$0), subjects again overestimate the probability of a black ball with little difference in errors between treatments with FP events only and with both FP and FN events. It seems that because the false-positive rate negatively affects the posterior, subjects fail to adjust their beliefs enough in response to FP rates.




\section{WTP and Signal Characteristics}\label{sec:results}

\subsection{Are Subjects Risk Neutral, Expected Utility Maximizers?} 

\begin{hypothesis}\label{hyp:eqRN} 
Subjects' WTPs for signals are equal to their value for risk-neutral agents. 
\end{hypothesis}

\begin{hypothesis}\label{hyp:eqSen} 
Subjects' preferences demonstrate equal sensitivity to costs generated by false-positive and false-negative events. 
\end{hypothesis}

\begin{result} 
On average, there are no significant discrepancies between WTP and predicted value for risk-neutral agents. When splitting by a signal type, the difference emerges only for signals with both false-positive and false-negative events.
\end{result}

\begin{result} 
On average for our signal and sample structure, we cannot reject the hypothesis of equal sensitivity. However, we observe significant heterogeneity with respect to priors: subjects tend to overvalue false-negative costs for low probability events and overvalue false-positive costs for high probability events. 
\end{result}





%The model provides a simple benchmark to evaluate potential welfare gains from the signal. Depending on the assumptions on signal characteristics and risk preferences, agents can pay more or less than the risk-neutral benchmark.
%
%We next examine whether subjects understand the WTPE task. One of the model's basic prediction is that signal value decreases with false positive and false negative rates. If subjects understand the basic premise of the WTPE, then we expect a negative correlation between the WTP and the signal's false positive and false negative rate. 

The theoretical value of a signal for a utility maximizing risk-neutral subject (hereafter, risk-neutral signal value) in equation~\ref{eq:rnWTP} provides a useful benchmark of our subjects' WTP. Figure~\ref{fig:WTPhist} plots the distribution of the differences between subjects' WTP and the signal value.  
%WTP is bounded between \$0 and \$5, where the latter is because protection can be purchased for \$5 so if information were that valuable, an individual would just choose to protect.  
The WTP is centered around the risk-neutral signal value, which is suggestive that subjects approximates a risk-neutral utility maximizing agent. However, there is substantial variation: only 25\% of actual WTP are within \$0.50 of the risk-neutral signal value, and subjects overvalue by at least \$1.5 in 22\% of cases and undervalue it by at least \$1.5 in 19\% of cases. Introducing FP and FN rates doesn't increase the range or variation of discrepancies, but introduces a long tail of positive discrepancies shifting the average up.


\begin{figure}[H]\centering 
	\caption{Discrepancy (Observed WTP - Signal value) by Signal Type} \label{fig:WTPhist}
	\subcaptionbox{All signals}{\includegraphics[width=.48\textwidth]{Graphs/hist_WTP_discr1.png}}
	\hfill
	\subcaptionbox{FP only}{\includegraphics[width=.48\textwidth]{Graphs/hist_WTP_discr1fp.png}}
	\hfill
	\vspace{2em}
	\subcaptionbox{FN only}{\includegraphics[width=.48\textwidth]{Graphs/hist_WTP_discr1fn.png}}
	\hfill
	\subcaptionbox{Both FP and FN}{\includegraphics[width=.48\textwidth]{Graphs/hist_WTP_discr1fpfn.png}}
	\hfill

\end{figure}

Our non-parametric analysis in Table~\ref{WTP_nonpar} also finds no differences between the (empirical) WTP and the risk-neutral signal value for 3 out of 4 broad categories of signal characteristics: honest signals; signals with only false-positive events; and signals with only false-negative events. When there are both false positives and false negatives, however, subjects significantly overvalue signals relative to the theoretical benchmark. These signals also tend to cause overprotection in the IP task.


%%%%%%%%%%%%%%%%%%%%%%%%%%%%%%%%%%%%%%%%%%%%%%%%%%%%%%%%%%%%%%%%%%%%%%%%%%%%%%%%%%%%%%%%%%%%%%%%%

\begin{table}[H]\centering \caption{Average WTP discrepancy (WTP-Value) by Signal Type} \label{WTP_nonpar} \begin{tabular}{cccc} \hline \hline
\multirow{2}{12ex}{\centering \textbf{False positive}}&\multirow{2}{12ex}{\centering \textbf{False negative}}&\multirow{2}{15ex}{\textbf{\centering Mean WTP discrepancy}}& \multirow{2}{10ex}{\centering \textbf{P($=0$)}}\\ 
\\
\hline
No&No&-0.106&0.433\\
No&Yes&0.143&0.250\\
Yes&No&0.081&0.502\\
Yes&Yes&0.492&0.000\\
\hline \end{tabular} \end{table}


%To understand how subjects' WTP deviates from these risk-neutral signal value, we use a regression analysis of discrepancies between reported WTP and theoretical values to understand, first, the validity of a risk-neutral model and, second, relative weights put on false-negative and false-positive costs. 

We estimate these relationships more formally with the following regression:
\[\Delta b_{is} = \beta_0 + \beta_1 FP + \beta_2 FN + \varepsilon_{is}\]
where $\Delta b_{is} = (b_{is} - b^*_s)$ is the difference between the WTP of individual $i$ for signal $s$ and $b^*_s$ is the signal value; FP (FN) is the false positive (false negative) cost. All specifications include subject fixed effects, with standard errors clustered at the subject level. If subjects are risk-neutral expected-utility-maximizing subjects, we expect $\beta_1=0$ and $\beta_2=0$. The result, reported in column 1 of Table~\ref{tab:wtp_ols}, shows positive and statistically significant coefficients for both FP and FN costs. In other words, subjects deviate from the benchmark model and overpay for inaccurate signals. 


\begin{itemize}
	\item\agt{XXX Is a false positive event = having a black gremlin in the group? Should we write false positive rate or is there a distinction you want to highlight?} \aut{We can describe them alternatively as environments with positive FP/FN rates, not sure if it makes it better.}
\end{itemize}


%\subsection{The (As-)symetry of False Positive and False Negative Signals} 

%First, subjects' WTPs for a signal are equal to its value for risk-neutral agents. The model provides a simple benchmark to evaluate potential welfare gains from the signal. Depending on the assumptions on signal characteristics and risk preferences, agents can pay more or less than the risk-neutral benchmark.
%
%Second, subjects' preferences value the marginal costs from false-negative and false-positive information equally. The model of a risk-neutral agent suggests that they should. Our derivations above indicate that the relative weight of false-negative costs can be either below or above one depending on risk preferences only.

%\begin{hypothesis}\label{hyp:eqRN} 
%Subjects' WTPs for a signal are equal to its value for risk-neutral agents. 
%\end{hypothesis}

%%%%%%%%%%%%%%%%%%%%%%%%%%%%%%%%%%%%%%%%%%%%%%%%%%%%%%%%%%%%%%%%%%%%%%%%%%%%%%%%%%%%%%%%%%%%%%%%%

\begin{table}[htbp!]
\centering
\adjustbox{max width = \textwidth}{
	\begin{threeparttable}
	\caption{Deviations from Signal Value (WTP - Value) and Signal Characteristics}
	\label{tab:wtp_ols}
	\begin{tabular}{l*{5}{c}}
		\hline\hline
		&\multicolumn{3}{c}{\multirow{2}{*}{All}} & \multicolumn{2}{c}{Prior}\\ \cmidrule(lr){5-6}
		&&&& $\{.1,.2\}$ & $\{.3,.5\}$\\ 
		\cmidrule(lr){2-4} \cmidrule(lr){5-5} \cmidrule(lr){6-6}  
		&\multirow{1}{*}{(1)} & \multirow{1}{*}{(2)} & \multirow{1}{*}{(3)} & \multirow{1}{*}{(4)} & \multirow{1}{*}{(5)}\\
	\hline
	\\ [-0.5em]
		\exInput{Tables/wtpdiff_ols.tex}
				\hline
		Subject FE & Yes & Yes & Yes & Yes & Yes \\
		Inaccurate Belief Interactions & No & No & Yes & Yes & Yes \\
		Prior Probability FE & No & No & No & Yes & Yes \\
		\hline\hline
	\end{tabular}
	\begin{tablenotes}[flushleft]
		\item\leavevmode\kern-\scriptspace\kern-\labelsep \small\textit{Notes:} */**/*** denotes 10/5/1 percent significance levels.
	\end{tablenotes}
	\end{threeparttable}
}
\end{table}		

		
%%%%%%%%%%%%%%%%%%%%%%%%%%%%%%%%%%%%%%%%%%%%%%%%%%%%%%%%%%%%%%%%%%%%%%%%%%%%%%%%%%%%%%%%%%%%%%%%%


The model of a risk-neutral agent suggests that subjects' preferences value the marginal costs of false-negative and false-positive events symmetrically. 
%Our derivations above indicate that the relative weight of false-negative costs can be either below or above one depending on risk preferences only. 
%Table~\ref{tab:wtp_ols} shows that on average subjects underreact to both FP and FN cost when choosing their willingness-to-pay as compared to a risk-neutral decision maker. If subjects were to behave consistently with our (oversimplified) model, the coefficients on FP and FN costs should equal zero. And yet, 
%At the same time, a negative and statistically significant constant suggests underpayment for honest signals. 
We find (Table~\ref{tab:wtp_ols}) that the coefficient on FN costs is slightly larger indicating higher sensitivity to FP costs, but we cannot reject the hypothesis that the two coefficients are equal. It means that on average and given our priors (!), the risk-neutral model still provides good guidance with choosing optimal trade-off between false-positive and false-negative costs. However later we note that this equivalency breaks down when considering specific priors. 

\subsection{Risk Preference and Belief Accuracy}

Our baseline estimation in column 1 indicates significant deviations from the model's predictions. Since the benchmark signal value is based on the optimal choice of a risk-neutral Bayesian updater, deviations can arise from at least two sources. First, Proposition~\ref{thm:riskAverse} suggests that risk preferences can influence the sensitivity of WTP to these signal characteristics. Second, systematic biases during the updating process can also lead to deviations. 

We find that risk preferences have limited explanatory power for sensitivity to signal characteristics. We use data from the BP game to categorize subjects by their risk preference. We classify all the subjects with internally consistent BP choices into three risk-preference categories: risk averse, risk neutral, and risk loving.\footnote{Subjects who switched from no protection to protection at exactly the cost-loss ratio $\pi=0.2$ are considered risk-neutral, while switching at lower (higher) levels indicates risk aversion (risk-loving). In addition, we created a dummy variable for subjects whose BP choices are inconsistent.}  Column 2 explores the heterogeneity of subject responses to FP and FN costs by their risk preference, with risk-neutral as the default category. The point estimates for risk-neutral subjects are about 1/3 smaller than those of the overall subjects (column 1) and statistically insignificant. This indeed indicates convergence towards prediction of a risk-neutral model, but with still sizable remaining discrepancies. Given the magnitudes of coefficients on the interaction of risk preferences with FP and FN costs, only risk-loving subjects have plausibly different sensitivities though the interaction terms are not statistically significant.

Accounting both for belief accuracy and for risk preferences also does little to explain the pattern of underreacting to FP and FN rates. To study the role of subjects' ability to Bayesian update, we use data from the BE task to categorize the WTP responses by belief accuracy.\footnote{We calculate a belief error as the absolute value of the difference between the subject's belief and the true posterior probability and then average these errors across all the decisions with identical priors, false positive and false negative rates. A subject's posterior belief for a decision is defined as accurate if its error is less than the median error across all the subjects making the same decision.} Column 3 presents the most flexible specification that controls for belief accuracy and risk preference by including triple interactions of belief accuracy, risk preference, and signal characteristics. The baseline group is the group of risk-neutral subjects with relatively accurate beliefs. We find that baseline subjects' sensitivity to FN costs declined (and remained insignificant), but their sensitivity to FP costs increased.
%Column 3 presents the heterogeneity analysis by belief accuracy. It suggests that inaccurate beliefs primarily lead to the undervaluation of an honest signal.

\iffalse Column 4 shows when based on an accurate belief, risk-neutral subjects' WTPs align well with the theoretical signal value for honest and false negative signals. Risk-neutral subjects are willing to pay a significant premium for false positive signals by a quite large amount --- around 0.7 to a dollar of the false positive cost. Our estimates are not precise enough to detect differences by risk preferences. However, the total coefficients for risk-averse and risk-loving subjects (presented at the bottom of the table) suggest that non-risk-neutral subjects responses to signal characteristics are opposite those who are risk neutral: they are unwilling to pay a premium for false positive signals, but willing to pay a premium for false negative signals. \fi

%These results suggest that, on average, subjects failed to fully account for signal quality, resulting in overpaying for signals with high false-positive and false-negative costs.  FP/FN cost significantly impact the deviation from the theoretical value no matter what else is included.  

\subsection{Heterogeneity by Prior}

We motivate our experiment with real world problems of designing signals for low-probability disasters. With a low prior, the default action of risk-neutral subject would be not to protect, and vice versa with a high prior. The signal would help risk-neutral subjects decide whether to keep the default action or to switch. We split the prior by below/above 0.25 (= protection cost/potential loss).  For estimation, we incorporated prior-probability fixed effects to the aforementioned flexible specification. 

\begin{itemize}
	\item\agt{Is this how we will interpret the results? As some kind of stickiness to the default action? We prime readers to expect that his would happen. Are we ruling other explanations? Should discuss! } \aut{Hit}
\end{itemize}

Column 4 presents the results for low-prior WTPE tasks. With a low prior, subjects overvalue false positive signals relative to the risk-neutral baseline. This overvaluation is similar for different risk preference profiles. In other words, with low priors, subjects overvalue signals that would induce them to overprotect. Both risk-neutral and risk-averse subjects do not (over-)value false-negative signals, while risk-loving subjects value such signals more positively --- with a difference that is statistically significant at 0.1 --- than risk-neutral subjects. The total coefficient of false-negative cost for risk-loving subjects is large, but is barely significant at 0.1 level. 

Column 5 presents the results for high-prior WTPE tasks. With a high prior, risk-neutral subjects report a WTP that aligns with the signal value. False positive signals are not overvalued, but both risk-averse and risk-loving subjects overvalue false negative signals. Our sample size cannot detect statistically significant differences in the extent of the overvaluation of false negative signals between both types and the risk-neutral subjects. These results imply a slight tendency (particularly for non-risk-neutral subjects) to overvalue signals that would induce them to underprotect.

To sum up, subjects overreact to false-negative rates and underreact to false-positive costs with low priors, but this pattern reverses for high priors. In practice, it implies that users would tend to overpay for alert signals with high false-positive costs, while excessively discounting systems with significant false-negative rates. For example, they would prefer a smoke alarm which never misses fires even if it involves higher expected costs of false alarms. Risk preferences have weak explanatory power: controlling for risk preferences reduces only the extra sensitivity of false-positive costs with priors but does not explain away the interaction between false-negative costs and priors. 

%While this finding is incidental and not the part of the original research plan, its significant practical implications prompt us to explore it further.



%XXX \agt{ALL: What would be nice if we can derive how higher prior would affect risk-averse/loving subjects. Also let's discuss what this really means and how to interpret. Does this mean that people overvalue signals that would change their default action?} \aut{It would be nice, but it does not seem currently possible in general case except the results already given in the theory section.} XXX

\subsection{Other Heterogeneity Analysis}

%There is also a possibility that the sensitivity is affected by subjects' demographic characteristics or their statistical education. 
We find little evidence for heterogeneous effects by demographic characteristics or statistical education. We show in (Appendix (\agt{?????}) Table~\ref{wtp_dem}) that gender or previous statistical education do not have significant effects on sensitivities for false-positive and false-negative rates. We find weak evidence that older (23+) subjects have a lower WTP for signals though the age variation is pretty limited. 
%Generally, the heterogeneity analysis is potentially constrained by  



\subsection{Discussion: What Accounts for the Heterogeneity by Prior?}

Many real-life examples of alerts such as fire alarms and mammograms have low priors. Hence it is important to explain the pattern of under-reaction to false-positive rates for low priors to see if this explanation extends to real life and if indicates any tools to manipulate this response.

Before going to explanations, we need to make sure that the pattern is robust and verify that it holds for the original WTP data and not only for differences with the benchmark value. In Table~\ref{tobit_het} we estimate directly the regression of reported willingness-to-pay on false-positive and false-negative costs given to subjects directly. We find that the sensitivities to both false-positive and false-negative rates increase with priors and that the change occurs relatively smoothly. Two sensitivities are also surprisingly close to each other.\footnote{For none of the priors can we reject the hypothesis that two sensitivities are equal to each other.} This drastically contrasts with the risk-neutral model, predicting that the sensitivity to FP rates should be much higher than the sensitivity to FN rates for low priors, but lower for high priors. This happens because for a given FN rate, false-negative events are much less likely with low priors and hence impose less costs on the agent. Increasing priors in theory makes FN rates more salient while the saliency of FP rates should go down. This discrepancy between WTP and the theoretical value explains changing signs on FP and FN costs in the previous regressions of WTP differences. Hence the puzzle can be reframed as the uniformity of WTP response to FP and FN rates for different priors.

\begin{table}[htbp]\centering
\def\sym#1{\ifmmode^{#1}\else\(^{#1}\)\fi}
\caption{WTP for Information, by prior (tobit)}
\begin{tabular}{l*{4}{c}}
\hline\hline
                &\multicolumn{1}{c}{(1)}&\multicolumn{1}{c}{(2)}&\multicolumn{1}{c}{(3)}&\multicolumn{1}{c}{(4)}\\
                &\multicolumn{1}{c}{0.1}&\multicolumn{1}{c}{0.2}&\multicolumn{1}{c}{0.3}&\multicolumn{1}{c}{0.5}\\
\hline
model           &                  &                  &                  &                  \\
FP rate         &    -2.91\sym{***}&    -2.08\sym{**} &    -4.46\sym{***}&    -3.25\sym{**} \\
                &    (1.1)         &    (1.0)         &    (1.0)         &    (1.3)         \\
FN rate         &    -2.48\sym{**} &    -2.73\sym{***}&     -3.7\sym{***}&    -3.65\sym{***}\\
                &    (1.1)         &    (1.0)         &    (1.0)         &    (1.3)         \\
Constant        &     1.79\sym{***}&     2.33\sym{***}&     2.71\sym{***}&     3.32\sym{***}\\
                &    (0.2)         &    (0.2)         &    (0.2)         &    (0.3)         \\
\hline
sigma           &                  &                  &                  &                  \\
Constant        &     1.83\sym{***}&      1.7\sym{***}&     1.72\sym{***}&     2.16\sym{***}\\
                &    (0.1)         &    (0.1)         &    (0.1)         &    (0.2)         \\
\hline
Observations    &      159         &      153         &      159         &      153         \\
Adjusted \(R^{2}\)&                  &                  &                  &                  \\
\hline\hline
\multicolumn{5}{l}{\footnotesize Standard errors in parentheses}\\
\multicolumn{5}{l}{\footnotesize \sym{*} \(p<0.10\), \sym{**} \(p<0.05\), \sym{***} \(p<0.01\)}\\
\end{tabular}
\end{table}
\label{tobit_het}

There are several candidate explanations for the observed pattern. Fist, we know that risk preferences affect sensitivities to FP and FN rates making them a natural candidate. Second, these choices are consistent with subjects neglecting the difference between false-positive and false-negative signals (black-eyed and white-eyed gremlins in the presentation) leading to similar coefficients on FP and FN rates and less changes of coefficients with priors. Third, the heterogeneity could be caused by subjects putting some value on non-instrumental information or signals that do not change their actions. Finally, the pattern can emerge due to some subjects failing to account for changes in prior probabilities between rounds (anchoring). 

We start with the most a priori probable explanation of risk preferences. Proposition 3 predicts that risk averse subjects have a stronger reaction to false-negative costs as compared to risk-neutral subjects. But our sample includes both risk-averse, risk-neutral, and risk-loving individuals, hence the average response does not necessarily follow any of the patterns consistent with uniform risk-aversion. 

We test the risk preferences explanation by using subjects' blind protection choices. If subjects express the same risk preferences across tasks, then accounting for their blind protection choices should either explain away the observed heterogeneity pattern or significantly reduce it. However we do not observe that. This is already obvious from Columns in Table with very different coefficient on FP and FN costs for low and high priors. In Table~\ref{wtp_het_risk} we augment this analysis by explicitly testing for interactions between risk-preferences, priors and false-positive and false-negative rates. We find that these interactions tend to be insignificant with the exception of interactions between false-negative rates and high risk aversion for some specifications. The heterogeneity largely remains after controlling for risk preferences, but the interaction between high priors and false-positive rates becomes insignificant.

\aut{There is also a possibility that risk preferences do not translate across different tasks, but still explain heterogeneity by priors in the WTP task. This would mean that each subject still reacts to costs of false-positive and false-negative events in the way consistent with the expected utility though reactions are different. In order to strike out this explanation we need to make sure that the answers are not rationalizable by any distribution of risk preferences, and there is a lot of degrees of freedom in risk preferences/utility functional forms. The most straightforward way to accomplish it is to show that each (or most) individual actions are not consistent with EU. I thought about testing it, but could not come up with any solution using our WTP data. }


\agt{XXX WHERE SHOULD THIS GO? SHOULD WE ORGANIZE AS ALTERNATIVE EXPLANATIONS AND SHOOT DOWN? OR A FOOTNOTE?}


\begin{itemize}
	\item\agt{Motivate why we dig deeper into het by prior.}
	\item\agt{At the end, we shot down the anchoring explanation. Do we want to make this sort of two hypotheses we are examining? Also, I feel that there can be a better way to describe our preferred explanation. }
\end{itemize}


The closeness of coefficient estimates for FP and FN rates for Table ~\ref{tobit_het} suggests one possible explanation for the observed pattern of sensitivities. If subjects neglect the difference between false-positive and false-negative risks when choosing their WTP, it would explain both coefficients' similarity and flatness of sensitivities with respect to priors. If subjects treat FP and FN rates the same and consider only the total proportion of false signals, they would assign equal weights to each of them. And because priors affect FP and FN costs in opposite ways, the best fit line of signal's value with the respect to the sum of FP and FN rates should be relatively flat. Note also that the equality of coefficients on FP and FN rates is a necessary prediction of this explanation, but can emerge only by chance with (some) heterogeneous risk preferences.

Subjects' verbal explanations of choices made in Informed Protection provide additional credence for the hypothesis that subjects bundle FP and FN rates together. For example, the explanation from one of the subjects states\textit{"I took into consideration how many honest there were and looked at the chances of picking a ball."}, while another also states \textit{"If there were only honest gremlins then I never protected but even if there was one white-swamp gremlin or one black-swamp gremlin then I payed for protection."}. In total, we find that 39 subjects out of 105 subjects in the main waves refer to the percentage of dishonest gremlins as their rationale for choosing protection. Many other could rely on this heuristic without giving a complete accurate explanation.

In order to test this hypothesis, we use choices from other tasks also using imperfect signals: Belief Elicitation and Informed Protection. If subjects systematically neglect the difference between false-positive and false-negative rates, we expect to find the pattern of abnormal reaction to FP and FN rates in cases when they do not affect the posterior. Namely, subjects would show sensitivity to FP rates when the signal is white and sensitivity to FN rates with black (positive) signals. This happens because some subjects react to FP (FN) rates as if they are FN (FP) rates. If present, this pattern cannot be explained by any distribution of risk preferences or by anchoring on previous priors.

First, we test this pattern for the BE choices by estimating the relationship between belief errors and FP and FN rates by signal type. Table~\ref{updateErrorReg} reports our results. We estimate a linear regression of updating error (actual posterior - reported belief) on FP and FN rates by signal color. We used fixed effects to control for individual updating biases. Consistent with our hypothesis, we observe that the FP rate has a significant positive effect on the error when the signal is white (negative), and that FN rate has a significant negative effect when the signal is black (positive). False-positive rates should not affect beliefs with white signals because a white signal (negative) can never be a false positive. The significance of FN rate for black signals is similarly an anomaly inconsistent with rational updating. 

%%%%%%%%%%%%%%%%%%%%%%%%%%%%%%%%%%%%%%%%%%%%%%%%%%%%%%%%%%%%%%%%%%%%%%%%%%%%%%%%%%%%%%%%%%%%%


\begin{table}[htbp]\centering 
\caption{Updating Errors in BE Task} 
\label{updateErrorReg}
\adjustbox{max width=\textwidth}{
	\begin{threeparttable}
	\begin{tabular}{l*{3}{c}}
	\hline\hline
									&\multirow{2}{8ex}{\centering All}&\multicolumn{2}{c}{Signal Received}\\ \cmidrule{3-4}
									&&\multicolumn{1}{c}{White}&\multicolumn{1}{c}{Black}\\
									&\multicolumn{1}{c}{(1)}&\multicolumn{1}{c}{(2)}&\multicolumn{1}{c}{(3)}\\
	\hline

	FP rate         &       .6\sym{***}&     .292\sym{***}&     .908\sym{***}\\
                &    (0.1)         &    (0.1)         &    (0.1)         \\
FN rate         &    .0108         &     .273\sym{***}&    -.251\sym{***}\\
                &    (0.1)         &    (0.1)         &    (0.1)         \\
Constant        &   -.0784\sym{***}&     .314\sym{***}&     -.47\sym{***}\\
                &    (0.0)         &    (0.0)         &    (0.0)         \\
Subject FE      &      Yes         &      Yes         &      Yes         \\
\hline
Observations    &     1248         &      624         &      624         \\
Adjusted \(R^{2}\)&     0.15         &     0.41         &     0.52         \\


	\\ [-1em]
	\hline
	Subject FE      &      Yes         &      Yes         &      Yes         \\
	\hline
	\hline\hline
	\end{tabular}
	\begin{tablenotes}[flushleft]
			\item\leavevmode\kern-\scriptspace\kern-\labelsep \footnotesize \textit{Notes:} Standard errors in parentheses. \sym{*} \(p<0.10\), \sym{**} \(p<0.05\), \sym{***} \(p<0.01\). 
	\end{tablenotes}								
	\end{threeparttable}
	}
\end{table}


%%%%%%%%%%%%%%%%%%%%%%%%%%%%%%%%%%%%%%%%%%%%%%%%%%%%%%%%%%%%%%%%%%%%%%%%%%%%%%%%%%%%%%%%%%%%%



%%%%%%%%%%%%%%%%%%%%%%%%%%%%%%%%%%%%%%%%%%%%%%%%%%%%%%%%%%%%%%%%%%%%%%%%%%%%%%%%%%%%%%%%%%%%



In Table~\ref{tab:protectReg}, we regress informed protection decisions on FP and FN rates and flexible controls of both posteriors and reported beliefs:\footnote{Given that the true functional form is unknown, we use a linear probability model to get unbiased coefficient estimates.}

	\[Prob(s_{ij}=1)=\alpha_i+\beta_1 FP +\beta_2 FN +Z(p_{ij})+Z(\mu_{ij})+\epsilon_{ij} \]
where $s_{ij}$ is the protection decision of subject $i$ in treatment $j$, $\alpha_i$ - subject FE, $P_{10}$, $P_{01}$ are FP and FN false positive and false negative rates and $Z(p_{ij}),Z(\mu_{ij})$ are the splines of corresponding variables $P_{ij}, \mu_{ij}$ to control for these variables in the flexible way. Each spline is a function $Z(x)$ which is just linear $x+C$ within one interval, and constant everywhere else. The splines are constructed so that their linear intervals cover the whole domain of probabilities and beliefs $[0,1]$.  \footnote{We use Stata mkspline command to create 5 splines $z_1(x),z_2(x),..z_5(x)$ of initial variable $x$ over the range $[0,1]$ such that $z_k(x)=\min[0,x-x_{k-1},x_k-x_{k-1}]$ with $x_k$ being equally spaced knot values. Splines account for potential nonlinear effects of posteriors and beliefs on protection decision with limited effect on degrees of freedom.} of posteriors and reported beliefs $\mu_{ij}$ for corresponding treatments. Columns 1 and 2 include only the flexible controls of the true posteriors. Columns 3 and 4 add further flexible controls to account for subjects' (often incorrect) estimates of the posterior, inferred from their BE responses.

Columns 1 and 2 show that even conditional on posterior and subject FEs to control for risk preferences, IP responses are still affected by FP and FN rates. For a white signal, FP and FN rates increase the tendency to overprotect; while for a black signal, FP rate had an opposite effect with comparable magnitude but without statistical significance. Hence the first prediction of FP/FN rate confusion hypothesis holds: FP rates increase protection when the signal is white conditional on the posterior. The effect holds if allowing for heterogeneity of sensitivities to FP/FN rates with respect to priors (Column 2). The coefficient on FN rate for black signals however is small in magnitude and statistically insignificant at 10\%. Adding flexible controls for subjects' beliefs reduces the coefficient magnitude on FP rate for white signals (Columns 3 and 4), but the coefficients still remains significant. This indicates that while belief partially contribute to these protection anomalies, they cannot explain them completely (possibly due to subjects re-evaluating their beliefs between tasks). 


%%%%%%%%%%%%%%%%%%%%%%%%%%%%%%%%%%%%%%%%%%%%%%%%%%%%%%%%%%%%%%%%%%%%%%%%%%%%%%%%%%%%%%%%%%%%%
\begin{table}[htbp]\centering 
\caption{Informed Protection Response} 
\label{tab:protectReg}
\adjustbox{max width=\textwidth}{
	\begin{threeparttable}	
	\begin{tabular}{l*{4}{c}}
	\hline\hline
									&\multicolumn{1}{c}{(1)}&\multicolumn{1}{c}{(2)}&\multicolumn{1}{c}{(3)}&\multicolumn{1}{c}{(4)}\\
	\hline
		
		FP rate x (S=White)&    0.895\sym{***}&    0.943\sym{***}&    0.525\sym{***}&    0.571\sym{***}\\
                & (10.011)         & (10.145)         &  (5.518)         &  (5.850)         \\
FN rate x (S=White)&    0.537\sym{***}&    0.532\sym{***}&    0.307\sym{**} &    0.299\sym{**} \\
                &  (3.709)         &  (3.631)         &  (2.139)         &  (2.048)         \\
p$>$0.2         &    0.039\sym{**} &    0.041\sym{**} &    0.024         &    0.029         \\
                &  (2.408)         &  (1.965)         &  (1.558)         &  (1.457)         \\
S=Black         &    0.531\sym{***}&    0.542\sym{***}&    0.383\sym{***}&    0.374\sym{***}\\
                &  (5.161)         &  (4.758)         &  (3.653)         &  (3.268)         \\
FP rate x (S=Black)&   -0.032         &    0.025         &   -0.065         &   -0.000         \\
                & (-0.158)         &  (0.123)         & (-0.330)         & (-0.001)         \\
FN rate x (S=Black)&    0.103         &    0.069         &   -0.005         &   -0.021         \\
                &  (1.398)         &  (0.860)         & (-0.055)         & (-0.229)         \\
FP rate x (p$>$0.2)&                  &   -0.081         &                  &   -0.085         \\
                &                  & (-1.066)         &                  & (-1.081)         \\
FN rate x (p$>$0.2)&                  &    0.088         &                  &    0.048         \\
                &                  &  (0.891)         &                  &  (0.521)         \\
\hline
N               &     2424         &     2424         &     2424         &     2424         \\
Pseudo R-squared&    0.505         &    0.505         &    0.538         &    0.539         \\
Log-likelihood  & -830.188         & -829.168         & -773.731         & -773.018         \\


	
	\\ [-1em]
	\hline
	Subject FE & Yes & Yes & Yes & Yes \\
	Flexible controls for: \\
	\hspace{1.5ex} Posterior & Yes & Yes & Yes & Yes \\
	\hspace{1.5ex} Beliefs & No & No & Yes & Yes \\	
	\hline\hline
	\end{tabular}
	\begin{tablenotes}[flushleft]
			\item\leavevmode\kern-\scriptspace\kern-\labelsep \footnotesize \textit{Notes:} Coefficients are average marginal effects. \emph{t}-statistics in parentheses. Standard errors are clustered at the subject level. \sym{*} \(p<0.10\), \sym{**} \(p<0.05\), \sym{***} \(p<0.01\). 
	\end{tablenotes}								
	\end{threeparttable}
	}
\end{table}


Finally, we believe that this pattern of sensitivities does not come from subjects' anchoring on priors or from preferences for non-instrumental information. Given that each subject goes through two sets of treatments with two different priors and the order of priors is fixed, anchoring remains a theoretical concern. However, we find that most subjects change their decisions when going from one prior to another (92 out of 104). The average belief error in the BE task is actually lower for the second set of priors rather than for the first showing that changing priors does not increase subjects' confusion. And most importantly, there is still uniformity in coefficient ratio even if we limit our attention only to the first priors in each sequence (0.1 or 0.2)\footnote{Depending on session, the first 3 WTP treatments use either the prior of 0.1 or 0.2 and hence there is no anchoring on the previous prior.}.  Similarly, while there is multiple evidence on humans valuing information not affecting their decisions (non-instrumental information), the value of this information should increase in false-negative rates for low priors in order to countervail a higher sensitivity predicted in theory. Having the value of non-instrumental information to increase with errors seems a priori implausible.

To sum up, we observe a striking uniformity in sensitivity of WTP to both false-positive and false-negative rates. This pattern is consistent with subjects neglecting the difference between false-positive and false-negative signals. This hypothesis is supported by subjects' explanation and with abnormal sensitivities to false-positive and false-negative rates in other treatments in which they do not affect posterior probabilities. We find no evidence that risk preferences or anchoring explain the heterogeneity pattern on its own. We will leave the discussion of practical implications of this finding for the conclusion.



 

\vspace{20pt}
%%%%%%%%%%%%%%%%%%%%%%%%%%%%%%%%%%%%%%%%%%%%%%%%%%%%%%%%%%%%%%%%%%%%%%%%%%%%%%%%%%%%%%%%%%%%%

\section{Conclusion}


\clearpage


\bibliography{Alerts2}


\appendix

\newpage
\section{Tables}


\begin{table}[h!]
\caption{Demographic Characteristics of Subjects} \label{summ_tab}
\begin{tabular}{l*{7}{c}}
\hline\hline
 &  \multicolumn{2}{c}{All} & \multicolumn{2}{c}{$p\in\{0.1,0.3\}$} & \multicolumn{2}{c}{$p\in\{0.2,0.5\}$} \\
\hline
 & N & \%  & N & \%  & N & \%  \\
\hline
Male         &    43 & 41  &  22  &  41 & 21  & 41  \\

Age$>$23yrs old     &    14 & 13  &  6  & 11  &8   & 16  \\

Students     &    88 & 84  &  46  & 85  &  42 &  82 \\

Had statistics classes      &    63 & 60  & 37   & 69  & 26  &  51 \\

Total     &    105 & 100  & 54   & 100   &  51 &  100 \\

\end{tabular}


\end{table}

\begin{table}[h!]
\caption{Error Decomposition} \label{belief_decomposition}
{
\def\sym#1{\ifmmode^{#1}\else\(^{#1}\)\fi}
\begin{tabular}{l*{6}{c}}
\hline\hline
                &\multicolumn{1}{c}{(1)}&\multicolumn{1}{c}{(2)}&\multicolumn{1}{c}{(3)}&\multicolumn{1}{c}{(4)}&\multicolumn{1}{c}{(5)}&\multicolumn{1}{c}{(6)}\\
                &\multicolumn{1}{c}{OLS}&\multicolumn{1}{c}{FE}&\multicolumn{1}{c}{OLS}&\multicolumn{1}{c}{FE}&\multicolumn{1}{c}{OLS}&\multicolumn{1}{c}{FE}\\
\hline
Prior           &     .246\sym{***}&     .202\sym{***}&     .175\sym{***}&     .191\sym{**} &      .14\sym{**} &    .0403         \\
                &    (5.5)         &    (4.0)         &    (3.1)         &    (2.5)         &    (2.3)         &    (0.6)         \\
Signal          &      .43\sym{***}&      .43\sym{***}&     .327\sym{***}&     .327\sym{***}&     .539\sym{***}&     .539\sym{***}\\
                &    (6.3)         &    (6.3)         &    (3.2)         &    (3.2)         &    (5.3)         &    (5.3)         \\
Good quiz $\times$ Prior&                  &                  &     .143\sym{*}  &    .0207         &                  &                  \\
                &                  &                  &    (1.7)         &    (0.2)         &                  &                  \\
Good quiz $\times$ Signal&                  &                  &     .193         &     .193         &                  &                  \\
                &                  &                  &    (1.4)         &    (1.4)         &                  &                  \\
Stat. class $\times$ Prior&                  &                  &                  &                  &     .162\sym{*}  &     .264\sym{***}\\
                &                  &                  &                  &                  &    (1.9)         &    (2.8)         \\
Stat. class $\times$ Signal&                  &                  &                  &                  &    -.166         &    -.166         \\
                &                  &                  &                  &                  &   (-1.2)         &   (-1.2)         \\
\hline
Observations    &      280         &      280         &      280         &      280         &      280         &      280         \\
Adjusted \(R^{2}\)&     0.31         &     0.31         &     0.33         &     0.32         &     0.32         &     0.32         \\
\hline\hline
\multicolumn{7}{l}{\footnotesize Decomposition works only for imperfect signals}\\
\multicolumn{7}{l}{\footnotesize \sym{*} \(p<0.10\), \sym{**} \(p<0.05\), \sym{***} \(p<0.01\)}\\
\end{tabular}
}

\end{table}

%\begin{table}[hbp!]
\begin{table}[htbp]\centering
\def\sym#1{\ifmmode^{#1}\else\(^{#1}\)\fi}
\caption{Informed Protection Response: logit with flexible control for posteriors}
\begin{tabular}{l*{4}{c}}
\hline\hline
                &\multicolumn{1}{c}{(1)}&\multicolumn{1}{c}{(2)}&\multicolumn{1}{c}{(3)}&\multicolumn{1}{c}{(4)}\\
                &\multicolumn{1}{c}{}&\multicolumn{1}{c}{}&\multicolumn{1}{c}{}&\multicolumn{1}{c}{}\\
\hline
FP rate         &    0.365\sym{***}&    0.472\sym{***}&    0.593\sym{***}&    0.573\sym{***}\\
                &    (3.3)         &    (3.4)         &    (4.0)         &    (3.7)         \\
FN rate         &    0.168\sym{*}  &    0.611\sym{***}&    0.150         &    0.565\sym{**} \\
                &    (1.8)         &    (2.8)         &    (1.5)         &    (2.5)         \\
p$>$0.2         &    0.026         &    0.066\sym{***}&    0.047\sym{*}  &    0.055\sym{*}  \\
                &    (1.5)         &    (2.8)         &    (1.8)         &    (2.0)         \\
S=Black         &    0.004         &    0.426\sym{**} &   -0.023         &    0.473\sym{**} \\
                &    (0.1)         &    (2.5)         &   (-0.3)         &    (2.4)         \\
FP rate x (S=Black)&                  &   -0.655         &                  &   -0.690         \\
                &                  &   (-1.4)         &                  &   (-1.5)         \\
FN rate x (S=Black)&                  &   -0.561\sym{**} &                  &   -0.608\sym{**} \\
                &                  &   (-2.1)         &                  &   (-2.2)         \\
FP rate x (p$>$0.2)&                  &                  &   -0.293\sym{**} &   -0.160         \\
                &                  &                  &   (-2.3)         &   (-1.2)         \\
FN rate x (p$>$0.2)&                  &                  &    0.084         &    0.264         \\
                &                  &                  &    (0.5)         &    (1.6)         \\
\hline
Observations    &     1248         &     1224         &     1224         &     1224         \\
Adjusted \(R^{2}\)&                  &                  &                  &                  \\
\hline\hline
\multicolumn{5}{l}{\footnotesize \textit{t} statistics in parentheses}\\
\multicolumn{5}{l}{\footnotesize Reporting average marginal effects, subject FE, errors are clustered by subject.}\\
\multicolumn{5}{l}{\footnotesize With flexible controls of posterior probability}\\
\multicolumn{5}{l}{\footnotesize \sym{*} \(p<0.10\), \sym{**} \(p<0.05\), \sym{***} \(p<0.01\)}\\
\end{tabular}
\end{table}
 \label{ip_tab}
%\end{table}

\begin{table}[htbp]\centering
\def\sym#1{\ifmmode^{#1}\else\(^{#1}\)\fi}
\caption{WTP minus Value of Information: demographic determinants}
\begin{tabular}{l*{9}{c}}
\hline\hline
                &\multicolumn{1}{c}{(1)}&\multicolumn{1}{c}{(2)}&\multicolumn{1}{c}{(3)}&\multicolumn{1}{c}{(4)}&\multicolumn{1}{c}{(5)}&\multicolumn{1}{c}{(6)}&\multicolumn{1}{c}{(7)}&\multicolumn{1}{c}{(8)}&\multicolumn{1}{c}{(9)}\\
                &\multicolumn{1}{c}{}&\multicolumn{1}{c}{}&\multicolumn{1}{c}{}&\multicolumn{1}{c}{}&\multicolumn{1}{c}{}&\multicolumn{1}{c}{}&\multicolumn{1}{c}{}&\multicolumn{1}{c}{}&\multicolumn{1}{c}{}\\
\hline
FP costs        &     .237\sym{*}  &     .283         &     .352\sym{*}  &     .117         &     .215         &     .248\sym{*}  &     .291\sym{**} &     .479\sym{**} &     .515\sym{**} \\
                &    (0.1)         &    (0.2)         &    (0.2)         &    (0.2)         &    (0.2)         &    (0.1)         &    (0.1)         &    (0.2)         &    (0.2)         \\
FN costs        &     .353\sym{***}&     .322\sym{***}&     .247\sym{***}&     .395\sym{***}&     .303\sym{***}&     .303\sym{***}&     .249\sym{***}&     .493\sym{***}&     .453\sym{***}\\
                &    (0.1)         &    (0.1)         &    (0.1)         &    (0.1)         &    (0.1)         &    (0.1)         &    (0.1)         &    (0.1)         &    (0.1)         \\
Male            &                  &    -.193         &    -.157         &                  &                  &                  &                  &                  &                  \\
                &                  &    (0.3)         &    (0.4)         &                  &                  &                  &                  &                  &                  \\
Male $\times$ FP costs&                  &    -.153         &    -.193         &                  &                  &                  &                  &                  &                  \\
                &                  &    (0.2)         &    (0.2)         &                  &                  &                  &                  &                  &                  \\
Male $\times$ FN costs&                  &    .0791         &     .114         &                  &                  &                  &                  &                  &                  \\
                &                  &    (0.1)         &    (0.1)         &                  &                  &                  &                  &                  &                  \\
Stat. class     &                  &                  &                  &     -.24         &    -.142         &                  &                  &                  &                  \\
                &                  &                  &                  &    (0.3)         &    (0.4)         &                  &                  &                  &                  \\
Stat. class $\times$ FP costs&                  &                  &                  &     .198         &     .124         &                  &                  &                  &                  \\
                &                  &                  &                  &    (0.3)         &    (0.3)         &                  &                  &                  &                  \\
Stat. class $\times$ FN costs&                  &                  &                  &   -.0834         &   -.0226         &                  &                  &                  &                  \\
                &                  &                  &                  &    (0.1)         &    (0.1)         &                  &                  &                  &                  \\
$>$23 yrs       &                  &                  &                  &                  &                  &    -.366         &    -.647\sym{*}  &                  &                  \\
                &                  &                  &                  &                  &                  &    (0.4)         &    (0.4)         &                  &                  \\
$>$23 yrs $\times$ FP costs&                  &                  &                  &                  &                  &   -.0679         &    .0238         &                  &                  \\
                &                  &                  &                  &                  &                  &    (0.3)         &    (0.3)         &                  &                  \\
$>$23 yrs $\times$ FN costs&                  &                  &                  &                  &                  &      .35         &     .277         &                  &                  \\
                &                  &                  &                  &                  &                  &    (0.2)         &    (0.2)         &                  &                  \\
Good quiz       &                  &                  &                  &                  &                  &                  &                  &     .544\sym{*}  &     .527         \\
                &                  &                  &                  &                  &                  &                  &                  &    (0.3)         &    (0.4)         \\
Good quiz $\times$ FP costs&                  &                  &                  &                  &                  &                  &                  &    -.468\sym{*}  &    -.437\sym{*}  \\
                &                  &                  &                  &                  &                  &                  &                  &    (0.3)         &    (0.3)         \\
Good quiz $\times$ FN costs&                  &                  &                  &                  &                  &                  &                  &     -.26\sym{*}  &    -.299\sym{**} \\
                &                  &                  &                  &                  &                  &                  &                  &    (0.1)         &    (0.1)         \\
Constant        &    -.207         &    -.126         &     .391         &   -.0579         &     .419         &    -.157         &     .397\sym{*}  &      -.5\sym{**} &    .0455         \\
                &    (0.2)         &    (0.2)         &    (0.3)         &    (0.3)         &    (0.4)         &    (0.2)         &    (0.2)         &    (0.3)         &    (0.3)         \\
Prior dummies   &       No         &       No         &      Yes         &       No         &      Yes         &       No         &      Yes         &       No         &      Yes         \\
\hline
Observations    &      624         &      624         &      624         &      624         &      624         &      624         &      624         &      624         &      624         \\
Adjusted \(R^{2}\)&     0.05         &     0.05         &     0.21         &     0.05         &     0.21         &     0.05         &     0.21         &     0.06         &     0.22         \\
\hline\hline
\multicolumn{10}{l}{\footnotesize Standard errors in parentheses}\\
\multicolumn{10}{l}{\footnotesize \sym{*} \(p<0.10\), \sym{**} \(p<0.05\), \sym{***} \(p<0.01\)}\\
\end{tabular}
\end{table}
 \label{wtp_dem}

\begin{table}[htbp]\centering
\def\sym#1{\ifmmode^{#1}\else\(^{#1}\)\fi}
\caption{WTP minus Value of Information, risk aversion and sensitivity to FP and FN costs}
\begin{tabular}{l*{5}{c}}
\hline\hline
                &\multicolumn{1}{c}{(1)}&\multicolumn{1}{c}{(2)}&\multicolumn{1}{c}{(3)}&\multicolumn{1}{c}{(4)}&\multicolumn{1}{c}{(5)}\\
                &\multicolumn{1}{c}{}&\multicolumn{1}{c}{}&\multicolumn{1}{c}{}&\multicolumn{1}{c}{FE}&\multicolumn{1}{c}{FE}\\
\hline
p$>$0.2         &   -.0942         &   -.0953         &    -.875\sym{**} &    -.113         &    -.884\sym{***}\\
                &    (0.2)         &    (0.2)         &    (0.3)         &    (0.2)         &    (0.3)         \\
FN costs        &    -.229\sym{*}  &    -.514         &    -.755         &    -.488         &    -.805\sym{*}  \\
                &    (0.1)         &    (0.4)         &    (0.5)         &    (0.5)         &    (0.5)         \\
p$>$0.2 $\times$ FN costs&     .716\sym{***}&     .836\sym{**} &     1.24\sym{***}&     .826\sym{**} &     1.27\sym{***}\\
                &    (0.1)         &    (0.3)         &    (0.3)         &    (0.4)         &    (0.3)         \\
FP costs        &     .558\sym{***}&     .506\sym{*}  &     .334         &     .492\sym{*}  &      .27         \\
                &    (0.1)         &    (0.3)         &    (0.3)         &    (0.3)         &    (0.2)         \\
p$>$0.2 $\times$ FP costs&    -.933\sym{***}&    -.758\sym{*}  &    -.189         &    -.734         &    -.191         \\
                &    (0.2)         &    (0.4)         &    (0.6)         &    (0.5)         &    (0.6)         \\
Risk-loving $\times$ p$>$0.2 $\times$ FN costs&                  &     .245         &    -.733\sym{**} &     .164         &    -.633\sym{*}  \\
                &                  &    (0.2)         &    (0.3)         &    (0.3)         &    (0.3)         \\
Risk-averse $\times$ p$>$0.2 $\times$ FN costs&                  &     .174         &    -.526         &     .125         &    -.498         \\
                &                  &    (0.2)         &    (0.3)         &    (0.3)         &    (0.3)         \\
No risk av. measure $\times$ p$>$0.2 $\times$ FN costs&                  &     .135         &    -.531         &     .158         &    -.655         \\
                &                  &    (0.2)         &    (0.5)         &    (0.3)         &    (0.4)         \\
Risk-loving $\times$ p$>$0.2 $\times$ FP costs&                  &    -.239         &    -.613         &     .204         &    -.459         \\
                &                  &    (0.4)         &    (0.6)         &    (0.8)         &    (0.7)         \\
Risk-averse $\times$ p$>$0.2 $\times$ FP costs&                  &    -.152         &    -.986         &    -.262         &    -.978         \\
                &                  &    (0.4)         &    (0.6)         &    (0.7)         &    (0.6)         \\
No risk av. measure $\times$ p$>$0.2 $\times$ FP costs&                  &     .158         &     -.92         &    -.453         &    -1.09         \\
                &                  &    (0.7)         &    (0.6)         &    (0.7)         &    (0.7)         \\
Full risk pref interactions&       No         &       No         &      Yes         &       No         &      Yes         \\
\hline
Observations    &      624         &      624         &      624         &      624         &      624         \\
Adjusted \(R^{2}\)&     0.08         &     0.07         &     0.06         &     0.41         &     0.41         \\
\hline\hline
\multicolumn{6}{l}{\footnotesize Standard errors in parentheses}\\
\multicolumn{6}{l}{\footnotesize \sym{*} \(p<0.10\), \sym{**} \(p<0.05\), \sym{***} \(p<0.01\)}\\
\end{tabular}
\end{table}
 \label{wtp_risk}
%\begin{table}[htbp]\centering
\def\sym#1{\ifmmode^{#1}\else\(^{#1}\)\fi}
\caption{WTP minus Value of Information: instructions understanding}
\begin{tabular}{l*{4}{c}}
\hline\hline
                &\multicolumn{1}{c}{(1)}&\multicolumn{1}{c}{(2)}&\multicolumn{1}{c}{(3)}&\multicolumn{1}{c}{(4)}\\
                &\multicolumn{1}{c}{Bad}&\multicolumn{1}{c}{Good}&\multicolumn{1}{c}{Bad}&\multicolumn{1}{c}{Good}\\
\hline
p$>$0.2         &    -.346         &     .128         &        0         &        0         \\
                &    (0.3)         &    (0.3)         &      (.)         &      (.)         \\
FN costs        &    -.037         &    -.392\sym{**} &     .223         &    -.131         \\
                &    (0.2)         &    (0.2)         &    (0.2)         &    (0.1)         \\
p$>$0.2 $\times$ FN costs&     .682\sym{***}&     .743\sym{***}&     .269\sym{*}  &     .344\sym{***}\\
                &    (0.2)         &    (0.1)         &    (0.1)         &    (0.1)         \\
FP costs        &      .66\sym{***}&     .465\sym{***}&     .591\sym{***}&     .413\sym{***}\\
                &    (0.2)         &    (0.1)         &    (0.2)         &    (0.2)         \\
p$>$0.2 $\times$ FP costs&    -.574\sym{**} &    -1.29\sym{***}&     -.19         &    -.968\sym{***}\\
                &    (0.3)         &    (0.2)         &    (0.2)         &    (0.2)         \\
Constant        &    -.279         &    .0686         &    .0568         &      .47\sym{*}  \\
                &    (0.3)         &    (0.2)         &    (0.3)         &    (0.3)         \\
Prior dummies   &       No         &       No         &      Yes         &      Yes         \\
\hline
Observations    &      288         &      336         &      288         &      336         \\
Adjusted \(R^{2}\)&     0.10         &     0.07         &     0.22         &     0.24         \\
\hline\hline
\multicolumn{5}{l}{\footnotesize Standard errors in parentheses}\\
\multicolumn{5}{l}{\footnotesize \sym{*} \(p<0.10\), \sym{**} \(p<0.05\), \sym{***} \(p<0.01\)}\\
\end{tabular}
\end{table}

%\begin{table}[htbp]\centering
\def\sym#1{\ifmmode^{#1}\else\(^{#1}\)\fi}
\caption{WTP for Information (different risk aversion)}
\begin{tabular}{l*{6}{c}}
\hline\hline
                &\multicolumn{1}{c}{(1)}&\multicolumn{1}{c}{(2)}&\multicolumn{1}{c}{(3)}&\multicolumn{1}{c}{(4)}&\multicolumn{1}{c}{(5)}&\multicolumn{1}{c}{(6)}\\
                &\multicolumn{1}{c}{$\theta=0$}&\multicolumn{1}{c}{$\theta=0.5$}&\multicolumn{1}{c}{$\theta=1.0$}&\multicolumn{1}{c}{$\theta=1.5$}&\multicolumn{1}{c}{$\theta=2.5$}&\multicolumn{1}{c}{Heterogeneous $\theta$}\\
\hline
FP costs        &     .295\sym{**} &     .322\sym{**} &     .316\sym{**} &     .271\sym{*}  &     .151         &      .29\sym{**} \\
                &    (0.1)         &    (0.1)         &    (0.1)         &    (0.1)         &    (0.1)         &    (0.1)         \\
FN costs        &     .243\sym{***}&     .346\sym{***}&      .46\sym{***}&     .559\sym{***}&      .69\sym{***}&     .343\sym{***}\\
                &    (0.1)         &    (0.1)         &    (0.1)         &    (0.1)         &    (0.1)         &    (0.1)         \\
Constant        &     .254         &    -.146         &    -.664\sym{***}&    -1.32\sym{***}&    -1.81\sym{***}&    -.411         \\
                &    (0.2)         &    (0.2)         &    (0.2)         &    (0.2)         &    (0.2)         &    (0.3)         \\
Prior dummies   &      Yes         &      Yes         &      Yes         &      Yes         &      Yes         &      Yes         \\
\hline
Observations    &      504         &      504         &      504         &      504         &      504         &      504         \\
Adjusted \(R^{2}\)&     0.21         &     0.27         &     0.28         &     0.32         &     0.37         &     0.19         \\
\hline\hline
\multicolumn{7}{l}{\footnotesize Standard errors in parentheses}\\
\multicolumn{7}{l}{\footnotesize \sym{*} \(p<0.10\), \sym{**} \(p<0.05\), \sym{***} \(p<0.01\)}\\
\end{tabular}
\end{table}

%\begin{table}[htbp]\centering
\def\sym#1{\ifmmode^{#1}\else\(^{#1}\)\fi}
\caption{WTP for Information (different risk aversion)}
\begin{tabular}{l*{5}{c}}
\hline\hline
                &\multicolumn{1}{c}{(1)}&\multicolumn{1}{c}{(2)}&\multicolumn{1}{c}{(3)}&\multicolumn{1}{c}{(4)}&\multicolumn{1}{c}{(5)}\\
                &\multicolumn{1}{c}{$\theta=0$}&\multicolumn{1}{c}{$\theta=0.5$}&\multicolumn{1}{c}{$\theta=1.0$}&\multicolumn{1}{c}{$\theta=1.5$}&\multicolumn{1}{c}{$\theta=2.5$}\\
\hline
FP costs        &     .213\sym{*}  &     .246\sym{**} &     .246\sym{**} &     .201\sym{*}  &    .0858         \\
                &    (0.1)         &    (0.1)         &    (0.1)         &    (0.1)         &    (0.1)         \\
FN costs        &     .246\sym{***}&     .348\sym{***}&      .46\sym{***}&     .556\sym{***}&     .687\sym{***}\\
                &    (0.1)         &    (0.1)         &    (0.1)         &    (0.1)         &    (0.1)         \\
Constant        &     .413\sym{**} &    .0134         &    -.505\sym{***}&    -1.16\sym{***}&    -1.66\sym{***}\\
                &    (0.2)         &    (0.2)         &    (0.2)         &    (0.2)         &    (0.2)         \\
Prior dummies   &      Yes         &      Yes         &      Yes         &      Yes         &      Yes         \\
\hline
Observations    &      744         &      744         &      744         &      744         &      744         \\
Adjusted \(R^{2}\)&     0.20         &     0.25         &     0.26         &     0.30         &     0.35         \\
\hline\hline
\multicolumn{6}{l}{\footnotesize Standard errors in parentheses}\\
\multicolumn{6}{l}{\footnotesize \sym{*} \(p<0.10\), \sym{**} \(p<0.05\), \sym{***} \(p<0.01\)}\\
\end{tabular}
\end{table}





%%%%%%%%%%%%%%%%%%%%%%%%%%%%%%%%%%%%%%%%%%%%%%%%%%%%%%%%%%%%
% APPENDIX
%%%%%%%%%%%%%%%%%%%%%%%%%%%%%%%%%%%%%%%%%%%%%%%%%%%%%%%%%%%%

%\newpage\clearpage
%\appendix
%\section*{\LARGE Appendix}
%\addcontentsline{toc}{section}{Appendix}
%
%\resumetocwriting
\newpage
\section{Proofs}
\small

\subsection{Proposition 1}
%\small
\begin{proof} If protection costs are low enough $c<\pi L$ than the risk-neutral decision-maker should always protect without a signal:
$$U=\max[\pi(Y-L)+(1-\pi)Y,Y-c]=Y-c$$

It means that a strictly risk-averse decision-maker with a utility function $u()$ should also protect:
$$\pi u(Y-L)+(1-\pi)u(Y)<u(\pi(Y-L)+(1-\pi)Y)=u(Y-c)$$

Then denote stochastic payoff with a signal as $X$ so that expected utility with a signal is $Eu(X-b)$ where $b$ is the willingness-to-pay solving:
$$Eu(X-b)=u(Y-c)$$
 Let $b_0$ be the willingness-to-pay for a risk-neutral decision-maker. By Jensen's inequality:
$$Eu(X-b_0)<u(EX-b_0)=u(Y-c)=Eu(X-b)$$

Because expected utility with a signal is a decreasing function of $b_0$ we obtain $b>b_0$. \end{proof} 

\subsection{Proposition 2}
%\small
\begin{proof}
Use the mean value theorem to rewrite the sensitivity as:
$${db\over dP_{01}}=-{\pi u'(\zeta)(L-c)\over E[MU]},\zeta \in \left(Y-c-b, Y-L-b\right)$$
Now let $X$ denote a random payoff of the agent with a signal. A risk-averse decision-maker puts a positive value on the signal only if its expected payoff is higher than the payoff with full protection: $EX>Y-c-b$. If an agent is imprudent ($u'''<0$) then $E[MU]\equiv E[u'(X)]<u'(EX)$. Next, $u'$ being a strictly increasing function and $EX>Y-c-b$: $u'(\zeta)>u'(Y-c-b)>u'(EX)$. Hence ${u'(\zeta)\over E[MU]}>1$ and ${db\over dP_{01}}<-\pi(L-c)$. 
\end{proof}
%\normalsize

However, risk aversion can both increase and decrease subject's sensitivity to false-positive rates depending on the utility function curvature and signal's characteristics. Intuitively, an expected marginal utility of a strongly risk-averse subject with a bad signal can be lower than the average slope of the utility function between ($Y-c-b$) and ($Y-b$) which reduces sensitivity to false-positive rates. It can also be higher if either the signal is good or the curvature is small. We can only say that it is very likely that for low protection costs and small priors $\pi$ (leading to no automatic blind protection) the ratio of sensitivities to FP rates over FN rates should be lower for risk-averse subjects. 


\subsection{Proposition 3}

\begin{proof}

The proof is approximate and relies on Taylor expansion to measure the effect of risk aversion on sensitivities to false-positive and false-negative rates. Start by rewriting the equilibrium condition for willingness-to-pay as the expected sum of utility differences:
\begin{equation}
\begin{split}
P(0,0) (u(Y-b)-u(Y))+p(0,1)(u(Y-b-L)-u(Y-L))+P(1,0)(u(Y-c-b)-u(Y))+\\+P(1,1)(u(Y-b-c)-u(Y-L))=0
\end{split}
\end{equation}

Here, $P(x,y)$ is a shorthand for the probability of an event that the signal equals $x$ and the state equals $y$. Next, we expand the utility differences of $u(Y-b)-u(Y), u(Y-c-b)-u(Y)$ as Taylor series around $Y$ and $u(Y-L-b)-u(Y-L)$ difference around $Y-L$ to get the following equation:\begin{equation}
\begin{split}
P(0,0) [u'(Y)(-b)+o(b)]+p(0,1)[u'(Y-L)(-b)+o(b)]+P(1,0)[u'(Y)(-c-b)+o(c+b)]+\\+P(1,1)[u(Y)-u'(Y)(b+c)+o(b+c)-u(Y-L)]=0
\end{split}
\end{equation}
Then we drop the terms $o(b), o(b+c)$ which we expect to be small enough to neglect to obtain:
\begin{equation}
\begin{split}
P(0,0)u'(Y)b+P(0,1)(u'(Y)+[u'(Y-L)-u'(Y)])b+P(1,0)u'(Y)(c+b)+\\+P(1,1)(-u'(Y)(b+c)-(u(Y-L)-u(Y))=0
\end{split}
\end{equation}
Now we can express the equilibrium (approximate) WTP $b$ as:
\[b={ P(1,1){(u(Y)-u(Y-L)) \over u'(Y)}-P(S=1)c \over D} \]
Where the denominator $D\equiv 1-P(0,1)\left({(u'(Y)-u'(Y-L))\over u'(Y)}\right)$. Now we remember that $P(1,1) \equiv \pi P_{11}=\pi (1-P_{01}), P(S=1)=\pi (1-P_{01})+(1-\pi)P_{10}$ and take derivatives of equilibrium (approximate) WTP $b$ with respect to false-positive and false-negative rates:
\[{db \over dP_{10}}=-{(1-\pi)c \over D} \]
\[{db \over dP_{10}}=-\pi \left[ {{(u(Y)-u(Y-L)) \over u'(Y)}-c \over D}-\left({P(1,1){(u(Y)-u(Y-L)) \over u'(Y)}-P(s=1)c \over D^2}\right) {(u'(Y)-u'(Y-L))\over u'(Y)} \right] \]
For a strictly risk-averse subject the sensitivity to false-positive rates should be lower than for a risk-neutral one because $u'(Y)-u'(Y-L)<0$ by decreasing marginal utility leading to $D>1$. The opposite is true for strictly risk-loving subjects. It is hard to say something more specific about the sensitivity to false-negative rates. 

Dividing the sensitivity to FN rate to the sensitivities of FP rate, we also obtain that this ratio is greater than 1 for strictly risk-averse subjects and less than one for strictly risk-loving ones. 

\[{db/dP_{01} \over db/dP_{10}}={\pi \over (1-\pi)} \left[{(u(Y)-u(Y-L)) \over u'(Y)}-c +{(P(1,1){(u(Y)-u(Y-L)) \over u'(Y)}-P(s=1)c) \over D}{(u'(Y)-u'(Y-L))\over u'(Y)}  \right] \]

Note that the corresponding equation for the risk-neutral decision-maker puts the ratio of sensitivities to:
\[{db/dP_{01} \over db/dP_{10}}= {\pi \over (1-\pi)} \left[L-c \right] \]

Hence the question of comparison of two ratios is equivalent to the question of the sign of the following inequality:
\[{(u(Y)-u(Y-L)) \over u'(Y)}+{(P(1,1){(u(Y)-u(Y-L)) \over u'(Y)}+P(s=1)c) \over D}{(u'(Y-L)-u'(Y))\over u'(Y)}><L\]
However note that the first component in the left-hand sum is already greater ${(u(Y)-u(Y-L)) \over u'(Y)}>L$ for any strictly risk-averse decision-maker by a mean value theorem. Risk aversion also makes the second component positive as  $u'(Y-L)-u'(Y)<0$ and $P(1,1){(u(Y)-u(Y-L)) \over u'(Y)}+P(s=1)c>P(1,1)L-P(s=1)c>0$ is also positive as it equal the expected savings from using a signal. Hence the LHS is greater than the RHS $L$ leading to the ratio of sensitivities to be greater than for a risk-neutral decision-maker. The same argument applied in reverse will show that for a strict risk-loving decision-maker the ratio of sensitivities will be lower. 
\end{proof}



\captionsetup[figure]{list=yes}
\captionsetup[table]{list=yes}

\renewcommand{\contentsname}{\vspace{-1em}}
%\tableofcontents

%\newpage\clearpage
\setcounter{table}{0}
\setcounter{figure}{0}
\setcounter{equation}{0}
\setcounter{footnote}{0}
\setcounter{section}{0}
\renewcommand\thesection{\Alph{section}}
\renewcommand\theequation{\thesection.\arabic{equation}}
\renewcommand\thetable{\thesection.\arabic{table}}
\renewcommand\thefigure{\thesection.\arabic{figure}}
\renewcommand\thesubsection{\Roman{subsection}}


%\section{Subject Decisions}\label{app:subtasks}

\pmt{I'm not going to futz with the sections right now, but I would make section 4 Subject Decisions, then have various subsections}
\pmst{We begin with a brief overview of subject behavior in the different tasks. }

%XXXX NEED TO REWRITE HERE XXXX}

Overall, we find that subject behavior across our tasks is largely ``sensible," i.e., subjects respond to increases in risk by protecting more, but deviates from the risk-neutral theoretical benchmark.  Below we investigate several standard possible explanations, including risk preferences and failures of Bayesian updating, before exploring how the characteristics of the signal influences protection decisions, beliefs, and WTP.

We follow that discussion with a regression analysis to explain subjects' WTP for signals of different qualities. Our regression results suggest that subjects' WTP deviated from those of a risk-neutral utility maximizing subject which was driven by their failure to fully account for signal quality when calculating their WTP. Furthermore, we find that these deviations remained after controlling for risk aversion or subjects' ability to perform Bayesian updating. 

%The final subsection explored theories that are consistent with these observed behaviors.



%\subsection{Overview: Are Subjects Bounded Rational?}
\subsection{Blind Protection}



\pmst{\textbf{Blind Protection.}}\pmst{Subjects' responses in the BP task are generally consistent with the expected utility framework.} Figure~\ref{fig:ProtResponse} plots the probability of protection decision against \pmst{prior}posterior probability of a black ball for the BP task, where the posterior is equivalent to the prior, and the IP task. On aggregate, subjects protect more with a higher probability of a negative outcome: only 13\% subjects protect when the probability of a black ball is 10\% in contrast to 70\% protecting when the probability is 30\%. 
%Moreover, the majority of subjects (54\%) have a probability threshold above which they always protect. About 30\% of subjects make at least one switch in which they protect for probability $P_1$ and do not protect for at least one probability $P_2>P_1$. 

At the individual level, BP responses are also largely sensible but  indicate significant heterogeneity in terms of risk aversion. For approximately 70\% of subjects (X/Y), the probability of choosing protection increases monotonically in posterior probability. The remaining 30\% make at least one switch from protecting to not protecting and back, which is inconsistent with EU maximization. Among these switchers, however, 83\% (24/39) skip only a single increment of the presented probability scale, suggesting an inattention error.\footnote{For comparison, this reference on the Holt and Laury (2002) instrument suggests the XX\% (YY\%) of subjects switch at least (at most) once.}  Risk-neutral subjects maximize their expected utility by protecting whenever the prior probability exceeds 0.25, which is the ratio of the protection cost (\$5) to the potential loss (\$20). In contrast, many of our subjects start protecting for lower probabilities of 0.1 or 0.2 indicating strict risk aversion. As a point of reference, switching at the probability 0.1 corresponds to CRRA risk aversion of $\theta=2$, while switching at 0.2 corresponds to $\theta=0.573$ (see \ref{ra_table} in Appendix).  \aut{I need to review the literature here to compare.} A smaller group of subjects makes choices consistent with risk loving by never protecting or protecting for the probability of 0.3. We use the total number of protection choices as a measure of subjects' risk aversion, but following Holt and Laury (2002) exclude subjects switching more than once. Most of our results do not condition on risk aversion and hence are not affected by this calculation. \pmt{I don't understand exactly what you mean by using the total to ``calculate" risk preferences.  In HL, the total corresponds to a range for the parameter of a CRRA utility function, but I think we are just using the total as a relative metric in our sample?  Also, we probably need to make sure that everything doesn't go totally bananas if we include the multiple switchers, then offer in a footnote to send those results to anyone who cares.}

\subsection{Informed Protection}
%\bigskip\noindent\textbf{Informed Protection/Response to Signal.}\ \ \
Protection decisions are also positively correlated across tasks for a given individual.
Subjects receive more information in the IP task than in the BP task, though using that information requires that subjects engage in Bayesian updating.  Figure~\ref{fig:ProtResponse} shows that, consistent with their behavior in the BP task, the share of subjects protecting in the IP task is increasing in the posterior probabilities. Roughly 28\% of subjects break monotonicity in their protection responses with respect to posterior probabilities\footnote{ They do not protect for some treatments with posterior probability $P$ while protecting for a posterior probabilty $P'<P$.} which roughly equals the percentage of non-monotonic responses for the BP task. At the individual level, we also observe that the total number of times subjects choose protection in the BP task significantly correlates with their likelihood to protect in the IP task conditional on posteriors, but this explains only a very small part ($<$1\%) of variation in the IP decisions\footnote{We use a LPM to estimate this relationship, and while the coefficient on the total number of protection choices is significant at 99\%, $R^2$ increases from 0.295 to 0.3.} 

Subjects' responses in the first two tasks suggest that conditional on the true probabilities, subjects protected more in the IP task compared to the BP task at low probabilities, \aut{even though the difference is not statistically significant}\pmt{What is the test? Proportion of protection decisions for probabilities less than X or something?}. These choices suggest that subjects' updated beliefs overshot the true posterior at low probabilities.  Unfortunately, we cannot compare their decisions for higher probabilities because we do not present choices with higher prior in the BP task.

%%%%%%%%%%%%%%%%%%%%%%%%%%%%%%%%%%%%%%%%%%%%%%%%%%%%%%%%%%%%%%%%%%%%%%%%%%%%%%%%%%%%%%%%%%%%%%%%%%%%%%%%%%%%%%%%%%%

\begin{figure}[H]
\centering
\caption{Average Protection Response} \label{fig:ProtResponse}
\begin{subfigure}[t]{.45\textwidth}
  \centering
\includegraphics[width=\textwidth]{Graphs/blind_prot_sta.png}
\end{subfigure}
\begin{subfigure}[t]{.45\textwidth}
  \includegraphics[width=\textwidth]{Graphs/ip_response_lpoly.png}
\end{subfigure}
%\begin{subfigure}[t]{.48\textwidth}
  %\centering
  %\includegraphics[width=\textwidth]{Graphs/ip_response_lpoly.png}
%
%\end{subfigure}
\end{figure}
\pmt{Eventually we need to add notes to the figures}
%%%%%%%%%%%%%%%%%%%%%%%%%%%%%%%%%%%%%%%%%%%%%%%%%%%%%%%%%%%%%%%%%%%%%%%%%%%%%%%%%%%%%%%%%%%%%%%%%%%%%%%%%%%%%%%%%%%

%\bigskip\noindent\textbf{Belief Elicitation.}\ \ \ 
\subsection{Belief Elicitation}
While the IP task gives us a sense for how subjects utilize signals in making protection decisions, we observe only whether or not they choose to protect, which conflates preferences with potential errors in updating posteriors.  The BP task gives provides insight into subjects' risk preferences, while the BE task allows us to better understand to what extent updating errors influence decisions.  

We define updating errors as the difference between the posterior and subjects' elicited belief on the posterior probability of a black ball for a given signal.  We plot the distribution of the updating errors in the left hand column of Figure~\ref{fig:BeliefUpdate}, while the right hand column provides a scatter plot of the elicited beliefs against the true posterior with a fitted line.\pmt{I would prefer to take the r-squared out of this figure and put it in the table note with the correlations for each panel}\aut{Duly noted, will be incorporated.}.
%Since subjects were only given signal characteristics and not true posteriors, their IP responses reflect, inter alia, their ability to infer the true posteriors from signal characteristics. (I'm not sure I like my sentence better)
Panel A of Figure~\ref{fig:BeliefUpdate} uses all elicited beliefs and suggests that, while errors occur, beliefs are still sensible. The distribution of updating errors is centered at 0, with roughly one-half (51\%) concentrated within +/- 0.1 interval around zero. Overall, the correlation between the elicited beliefs and the true posteriors was 0.653.  

Using all the observations, however, obscures an important distinction: in many cases the ball color is completely certain based on priors and signals and so the updating should be trivial.  Panel B of Figure~\ref{fig:BeliefUpdate} includes only those 44\% beliefs elicited for an uncertain posterior. The median error is now -0.12, with with 90\% of errors between -0.48 and 0.3, suggesting that subjects tend to overestimate the likelihood of adverse events for uncertain posteriors, \pmt{which is consistent with what we see in Figure~\ref{fig:ProtResponse}}. The correlation between beliefs and posteriors in this subset of observations is only 0.571.  Panel C of Figure~\ref{fig:BeliefUpdate} plots the distribution of updating errors with certain posteriors, which includes: (i) treatments with all-honest gremlins; and (ii) treatments with obviously irrelevant dishonest gremlins (e.g., a group with honest and white-eyed gremlins with a hint that the ball is black --- or vice versa). Reassuringly, 69\% of reported beliefs are correct, but subjects still err in about 30\% of cases. About half of these errors involve reporting a probability of between one and zero, with the other half reporting a probability of one when it should have been zero. \aut{There is little evidence of these drastic errors (0 instead of 1) being strongly correlated with randomness in other treatments.} \pmt{It depends a bit.  Are some guys doing everything right here and others doing everything wrong?  Is there any relationship between errors here and inconsistencies in BP or a lower correlation between BP and IP tasks?} 

Overall, the pattern of belief updating is consistent with previous literature which finds that while humans usually update beliefs in a correct direction, they tend to underreact both to priors and the signals. The effect of underweighting priors, first noted in the psychology literature \citep*{phillips_conservatism_1966-1, tversky_belief_1971, kahneman_subjective_1972}, and is known under the names of representativeness bias or base rate neglect. Subjects sensitivity both to priors and signals is easy to measure through estimating the following equation ((first introduced by \citet{grether_bayes_1980}) which links the posterior probabilities $\mu(B|S)$ of the state $B$ conditional on signal $S$ with the prior log-odds $\log\left({P(S|B) \over P(S|W)}\right)$ of the signal and signal log-odds $\log \left({P(B)\over P(W)}\right)$: 
\begin{equation}
\log\left({\mu(B|S) \over 1-\mu(B|S)}\right)=\alpha \log\left({P(S|B) \over P(S|W)}\right)+\beta \log \left({P(B)\over P(W)}\right)
\end{equation}

Coefficients $\alpha$ and $\beta$ should equal one if subjects perfectly follow Bayesian updating. Our estimates of these parameters are significantly below one with $\hat \alpha=0.43$ $\hat \beta=0.25$ (see Column 1 in \ref{belief_decomposition}). This is consistent with the meta-analysis in \citet{benjamin_chapter_2019} which calculates the average $\hat \alpha$ estimate to be around 0.22 (0.4 for incentivized studies only) and the average $\hat \beta$ to be 0.6 (0.43 for incentivized) for studies presenting signals simultaneously (consistent with this study)\footnote{The common name for this kinds of experiments is \textit{bookbag-and-poker-chip experiments}}. Hence our subjects also demonstrate both the base-rate neglect and the signals underweighting. These effects lower the correlation between posteriors and reported beliefs and reduce sensitivity of beliefs to signal characteristics.  


%%%%%%%%%%%%%%%%%%%%%%%%%%%%%%%%%%%%%%%%%%%%%%%%%%%%%%%%%%%%%%%%%%%%%%%%%%%%%%%%%%%%%%%%%%%%%%%%%%%%%%%%%%%%%%%%%%%

\begin{figure}[H]
	\centering
	\caption{Errors in Bayesian Updating} \label{fig:BeliefUpdate}
	\subcaptionbox{Error Distribution}{\includegraphics[width=.48\textwidth]{Graphs/hist_belief_error_s3.png}}
	\hfill
	\subcaptionbox{Error v. Posterior}{\includegraphics[width=.48\textwidth]{Graphs/updating_s3.png}}
	\hfill
	\subcaptionbox{Error Distribution, Uncertain Color}{\includegraphics[width=.48\textwidth]{Graphs/hist_belief_error_s4.png}}
	\hfill
	\subcaptionbox{Error v. Posterior, Uncertain Color}{\includegraphics[width=.48\textwidth]{Graphs/updating_s4.png}}
	\hfill
	\subcaptionbox{Error Distribution, Certain Color}{\includegraphics[width=.48\textwidth]{Graphs/hist_belief_error_s5.png}}
	\hfill
	\subcaptionbox{Error v. Posterior, Certain Color}{\includegraphics[width=.48\textwidth]{Graphs/updating_s5.png}}
	\hfill
\end{figure}
\clearpage

%%%%%%%%%%%%%%%%%%%%%%%%%%%%%%%%%%%%%%%%%%%%%%%%%%%%%%%%%%%%%%%%%%%%%%%%%%%%%%%%%%%%%%%%%%%%%%%%%%%%%%%%%%%%%%%%%%%

%\bigskip\noindent\textbf{Willingness-to-Pay Elicitation.}\ \ \ 
%\subsection{Willingness-to-Pay Elicitation}
%The BP, IP, and BE tasks allow us to discern how individuals use freely provided information to make protection decisions, but signals are not necessarily provided freely.  To better understand how subjects value the signals, we elicit their Willingness-to-Pay for signals of differing degrees of informativeness.  Figure XX plots the distributions of the theoretically optimal WTP\pmt{We should explain here where this comes from} (Panel A), the distribution of actual WTP (Panel B), and the distribution of the discrepancies between the theoretical and actual WTP that we observe.  WTP is bounded between \$0 and \$5, where the latter is because protection can be purchased for \$5 so if information were that valuable, an individual would just choose to protect.  \pmt{Placeholder for where we say something about my suggested Panels A and B.}  Reported WTP is centered around the theoretical WTP, suggesting that it is a useful benchmark. However, the variation is very substantial. Only 25\% of actual WTP are within \$0.50 if the theoretical one, and subjects overvalue the signal by at least \$1.5 in 22\% of cases and undervalue it  by at least \$1.5 in 19\% of cases. 
%
%\begin{figure}[H]\centering 
%\caption{Willingness-to-Pay Distribution} \label{fig:WTPhist}
%\subcaptionbox{Value Distribution}{\includegraphics[scale=0.2]{Graphs/hist_value.png}}
	%\hfill
%\subcaptionbox{WTP distribution}{\includegraphics[scale=0.2]{Graphs/hist_WTP.png}}
	%\hfill
%\subcaptionbox{Discrepancy (WTP-Value)}{\includegraphics[scale=0.2]{Graphs/hist_WTP_discr1.png}}
	%\hfill
%\end{figure}



\end{document}